\documentclass[a4paper]{scrartcl}
\usepackage[utf8]{inputenc}
\usepackage[ngerman]{babel}
\usepackage{textcomp}
\usepackage{amsmath}
\usepackage{stmaryrd}

% zus�tzliche mathematische Symbole, AMS=American Mathematical Society 
\usepackage{amssymb}

% f�rs Einbinden von Graphiken
\usepackage{graphicx}

% f�r Namen etc. in Kopf- oder Fu�zeile
\usepackage{fancyhdr}

% erlaubt benutzerdefinierte Kopfzeilen 
\pagestyle{fancy}

% Definition der Kopfzeile
\lhead{
\begin{tabular}{ll}
Logik für Studierende der Informatik
\end{tabular}
}
\chead{}
\rhead{\today{}}
\lfoot{}
\cfoot{Seite \thepage}
\rfoot{} 
\parindent 0pt

\begin{document}
\tableofcontents
\newpage
\section{Aussagenlogik}

\subsection{Grundbegriffe}
\textbf{Defintion:} (Syntax) Eine aussagenlogische Formel ist eine Zeichenkette, die sich aus den Variablen $A_0, A_1,$ ... und den Symbolen $(, ), \wedge, \vee, \neg$ zusammensetzt und folgende Regeln einhält:
\begin{itemize}
\item Variablen sind Formeln 
\item Wenn $F_1$ und $F_2$ Formeln sind, dann ist $(F_1 \wedge F_2)$ eine Formel (\textit{Konjunktion})
\item Wenn $F_1$ und $F_2$ Formeln sind, dann ist auch $(F_1 \vee F_2)$ eine Formel (\textit{Disjunktion})
\item Wenn F eine Formel ist, ist auch $\neg F$ eine Formel (\textit{Negation})
\end{itemize}
Alle Formeln enstehen auf diese Weise.\\
\textbf{Beispiel:} $(\neg A_0 \wedge A_1)$\\
\textbf{Lemma:} (Eindeutige Lesbarkeit) Jede Formel ist
\begin{itemize}
\item eine Variable
\item eine Konjunktion $(F_1 \wedge F_2)$
\item eine Disjunktion $(F_1 \vee F_2)$
\item eine Negation $\neg F$
\end{itemize}
Es tritt immer genau einer dieser Fälle ein, und ggf sind $F_1$ und $F_2$, bzw. $F$ eindeutig bestimmt.\\
\textbf{Beispiel:} Wir bezeichnen das erste Zeichen der gegebenen Formel. Es treten die Fälle auf:
\begin{itemize}
\item es ist eine Variable $A_i$. Dann ist $A_i$ nach Definition die ganze Formel.
\item es ist das Zeichen ''$\neg$'', dann ist nach Definition unserer Formel $\neg F$. $F$ ist also der Rest unserer Zeichenkette.
\item es ist ''$($''. Nach Definition ist unsere Formel vom Typ $(F_1 \wedge F_2)$ / $(F_1 \vee F_2)$.
\end{itemize}
Wir nehmen an, dass wir die Formel außerdem auch als  $(F'_1 \wedge F'_2)$ / $(F'_1 \vee F'_2)$ lesen können. Dann ist entweder $F'_1$ Anfangsstück von $F_1$ oder umgekehrt. Nach dem folgenden Hilfssatz gilt $F_1 = F'_1$ und das unmittelbar folgende Zeichen muss ''$\wedge$'' oder ''$\vee$'' sein und legt den Typ der Formel fest. Wenn man eine Formel nach obigem Lemma zerlegt hat und die Formel keine Variable war, dann wendet man das Lemma anschie{\''ss}end weiter auf $F_1$ und $F_2$ bzw. $F$ an bis die Formel komplett zerlegt wurde.\\
\textbf{Hilfssatz:} Kein echtes Anfangsstück einer Formel ist selbst eine Formel. Mit ''\textit{echtem Anfangsstück}'' meinen wir ein Anfangsstück, das nicht die ganze Formel ist.\\
\textbf{Beispiel:} $(F_1$ ist Anfangsstück von $(F_1 \wedge F_2)$.\\
\textbf{Beweis:} Durch Induktion über die Länge der Formel. Sei $F$ Formel der Länge 1, dann hat $F$ das echte Anfangsstück '' '' (\textit{leere Formel}), aber das ist keine Formel.\\ 

\newpage

Wir nehmen jetzt an, dass der Hilfssatz für alle kürzeren Formeln bewiesen werden kann. Wir wenden eine Fallunterscheidung nach dem ersten Buchstaben an.
\begin{itemize}
\item Variable $A_1$: Dann hat $F$ die Länge 1.
\item ''$\neg$'': Nach Definition gilt: $F = \neg F_1$. Sei $F'$ echtes Anfangsstück von $F_1$, dann folgt $F' = \neg F_1$, und $F'_1$ ist echtes Anfangsstück von $F_1$. Nach Induktionsvoraussetzung ist aber kein echtes Anfangsstück von $F_1$ eine Formel. Also kann $F'$ keine Formel sein.
\item ''$($'': Wie im obigen Beweis gilt dann $F = (F_1 \wedge F_2)$ / $(F_1 \vee F_2)$. Sei $F'$ echtes Anfangsstück von $F$, dann folgt $F' = (F_1 \wedge F_2)$ / $(F_1 \vee F_2)$. Da $F_1$ und $F_2$ kürzer als $F$ sind und $F_1$ Anfangsstück von $F'_1$ oder $F'_1$ Anfangsstück von $F_1$ ist, muss $F_1 = F'_1$ gelten. Dann muss $F'_2)$ ein Anfangsstück von $F_2)$ sein, also $F'_2$ ein Anfangsstück von $F_2$. Nach Induktionsvoraussetzung kann $F'_2$ kein echtes Anfangsstück gewesen sein.
\end{itemize}
Damit wäre der Hauptsatz bewiesen. $\square$\\
\textbf{Definition:} Es sei $\mathcal{D}$ eine Menge von Variablen, dann ist eine Belegung eine Abbildung $\mathcal{A}$:
\begin{center}
$\mathcal{A}: \mathcal{D} \rightarrow \{0, 1\}$
\end{center}
Dabei stehen 0, 1 für die Wahrheitswerte ''\textit{falsch}'' oder ''\textit{wahr}''. Sei also $A_3 \in \mathcal{D}$ und $\mathcal{A}(A_3) = 1$, dann hat $A_3$ unter der Belegung $\mathcal{A}$ den Wahrheitswert 1 (''\textit{wahr}''). Wir definieren Verknüpfungen von Wahrheitswerten:

\[ w_1 \wedge w_2 = \left\{
  \begin{array}{l l}
    1 & \quad \text{$w_1 = 1$ und $w_2 = 1$}\\
    0 & \quad \text{$w_1 = 0$ oder $w_2 = 0$}
  \end{array} \right.\]
\[ w_1 \vee w_2 = \left\{
  \begin{array}{l l}
    1 & \quad \text{$w_1 = 1$ oder $w_2 = 1$}\\
    0 & \quad \text{$w_1 = 0$ und $w_2 = 0$}
  \end{array} \right.\]
\[ \neg w_1 = \left\{
  \begin{array}{l l}
    1 & \quad \text{$w = 0$}\\
    0 & \quad \text{$w = 1$}
  \end{array} \right.\]
\textbf{Definition:} (Semantik) Sei $\mathcal{A}$ eine Belegung der Variablen einer Formel $F$, dann hat $F$ den Wahrheitswert $\mathcal{A}(F)$ von $F$:
\begin{itemize}
\item $\mathcal{A}(A_1)$, falls $F = A_1$ Variable ist
\item $\mathcal{A}(F_1) \wedge \mathcal{A}(F_2)$, falls $F = (F_1 \wedge F_2)$
\item $\mathcal{A}(F_1) \vee \mathcal{A}(F_2)$, falls $F = (F_1 \vee F_2)$
\item $\neg \mathcal{A}(F')$, falls $F = \neg F'$
\end{itemize}

\newpage

Wegen des obigen Lemma ist $\mathcal{A}(F)$ dadurch wohldefiniert. Wir verwenden folgende Abkürzungen:
\begin{itemize}
\item $(F_1 \rightarrow F_2)$ für $(\neg F_1 \vee F_2)$
\item $(F_1 \Leftrightarrow F_2)$ für $((F_1 \rightarrow F_2) \wedge (F_2 \rightarrow F_1))$
\item $\bigwedge_{i < n}{F_i}$ für ($F_1 \wedge ... (F{n-2} \wedge F{n-1}) ... )$
\item $\bigvee_{i < n}{F_i}$ für ($F_1 \vee ... (F{n-2} \vee F{n-1}) ... )$
\item $\top$ (''\textit{wahr}'') für $(\neg A_0 \vee A_0)$
\item $\bot$ (''\textit{falsch}'') für $(\neg A_0 \wedge A_0)$
\end{itemize}
\textbf{Konventionen:} Wir können $\top$, $\bot$ auch als ''\textit{nicht zusammengesetzte}'' Formeln betrachten
\begin{itemize}
\item $\bigwedge\limits_{i < 0}{F_i} = \top$
\item $\bigvee\limits_{i < 0}{F_i} = \bot$
\end{itemize}
Außerdem können wir $(F_1 \vee F_2)$ als Abkürzungen für $\neg(\neg F_1 \wedge \neg F_2)$ auffassen. Gelegentlich lassen wir Klammern weg, wenn keine Missverständnisse auftreten. Unsere Abkürzungen haben genau wie alle Formeln Wahrheitswerte unter vorgegebenen Bedingungen.\\
\begin{center}
$\mathcal{A}(\top) = 1, \mathcal{\bot} = 0,$ usw.
\end{center}
Sei $\mathcal(A)$ eine Belegung der Variablen von $F$. Wenn $\mathcal{A}(F) = 1$, sagen wir, ''\textit{$\mathcal{A}$ erfüllt $F$}'', ''\textit{$\mathcal{A}$ ist Modell von $F$}'', ''\textit{$\mathcal{A} \models F$}''.\\
\textbf{Definition:} Eine Formel heißt ''\textit{allgemeingültig}'' oder ''\textit{Tautologie}'', wenn $\mathcal{A}(F) = 1$ für alle möglichen Belegungen gilt. Eine Formel $F$ heißt ''\textit{erfüllbar}'', wenn es eine Belegung $\mathcal{A}$ mit $\mathcal{A}(F) = 1$ gibt.\\
\textbf{Beispiel:}
\begin{itemize}
\item $\top = (\neg A_0 \vee A_0)$ ist allgemeingültig, $A_0$ ist erfüllbar.
\item $\bot = (\neg A_0 \wedge A_0)$ ist nicht erfüllbar. $A_0$ ist nicht allgemeingültig.
\end{itemize}
\textbf{Lemma:} 
\begin{itemize}
\item $F$ ist genau dann eine Tautologie, wenn $\neg F$ nicht erfüllbar ist.
\item $F$ ist erfüllbar genau dann, wenn $\neg F$ keine Tautologie ist.
\end{itemize}
\subsection{Äquivalenz von Formeln und Normalformen}
\textbf{Def.}:Zwei Formeln F,G in den gleichen Variablen heißen \underline{äquivalent}, kurz $F\equiv G$, wenn $\mathcal{A}(F)=\mathcal{A}(G)$ für alle Belegungen der Variablen.\\
\textbf{Beispiel}: F = A, G = A$\land$(A$\lor$B)\\
\textbf{Bemerkung}: \begin{itemize}
\item F$\equiv G$ genau dann, wenn $F\leftrightarrow G$ allgemeingültig ist.\\ 
\item F allgemeingültig (kurz $\models$F) genau dann, wenn $F\equiv \top$\\
\item F erfüllt (kurz $\nvDash \neg $ F) genau dann, wenn F$\not\equiv \bot$\\
\item "$\equiv$" ist eine  Äquivalenzrelation. [d.h. F$\equiv$F für alle Formeln F\\
F$\equiv G \Rightarrow G \equiv F$ für alle Formel n F,G\\
(F$\equiv G $ und $G\equiv H)\Rightarrow F \equiv H$ für alle Formeln F,G,H\\
Das bedeutet: Die Menge aller Foemeln zerfällt in Äquivalnzklassen [F] mit G$\in$[F]$\Leftrightarrow F\equiv G$ ]\\ 
\end{itemize}
\textbf{Bemerkung}: Methoden um Äquivalenz von Formeln zu zeigen:\\
\begin{itemize}
\item Fallunterscheidungen für die Belegungen (siehe Bsp, mündlich)\\
\item Alle Belegungen einsetzen und vergleichen\\
(A$\land B ) \equiv (B\land A)$ \\
\begin{tabular}{l|l|l}
$_A\backslash ^B$ & 0 & 1\\ \hline 0 & 0 & 0\\ \hline 1 & 0 & 1\\\end{tabular}
\begin{tabular}{l|l|l}
$_A\backslash ^B$ & 0 & 1 \\ \hline 0 & 0 & 0 \\ \hline 1 & 0 & 1 \end{tabular}\\
\item Venn-Diagramme: Für jede Variable A male eine "Menge" $M_A$. Punkte in $M_A$ entsprechen Belegungen mit $\mathcal{A}(A)=1$, Punkte außerhalb:$\mathcal{A}(A)=0$\\
\textbf{Bsp}: (($A\land B)\lor C ) \equiv ((A\lor C ) \land (B \lor C))$\\
\textbf{Hier noch Grafik einfügen !!!}\\
\end{itemize}
\textbf{Def.} Sei F eine Formel, aufgebaut aus Variablen, den Junktoren $\neg, \land, \lor,$ sowie $\top,\bot$. Die zu F \underline{duale} Formel F$^*$ entsteht aus F durch Vertauschen von $\land, \lor$ sowie von $\top, \bot$.
\textbf{\underline{Lemma}}:F$\equiv$G genau dann, wenn F$^* \equiv$G\\
\textbf{\underline{Beweis}}: Es sei $\mathcal{A}$ eine Belegung . Vertausche in $\mathcal{A}$ die Werte 0,1 und erhalte eine neue Belegung $\mathcal{A}^*$ \\
\textbf{Behauptung}: es gilt $\mathcal{A}(F)=1$ genau dann, wenn $\mathcal{A}^*(F^*)=0$.\\
\textbf{Begründung}: Durch Induktion über Länge der Formel mit dem Lemma über eindeutige Lesbarkeit.\\
$$F = A\Rightarrow F^*=A$$\\
$$ \mathcal{A}(F) = \mathcal{A}(A)=1 \Leftrightarrow \mathcal{A}^*(A)=0=\mathcal{A}^*(F^*))$$\\
$$F=\top \Rightarrow F^*=\bot$$\\
$$\mathcal{A}(\top) = 1\text{gilt immer genau} A^*(\bot)=0$$\\
Analog für F=$\bot$\\
Induktionsschritt:\\
\begin{itemize}
\item F = $\neg$ G : \\
$$ \mathcal{A}(F) = 1 \Leftrightarrow \mathcal{A}(G) = 0$$
$$\overset{IV}{\Leftrightarrow} \mathcal{A}^*(G^*)=1 \Leftrightarrow \mathcal{A}^*(F^*)=0$$\\
\item F = ($F_0 \land F_1) , F^*=(F_0^*\lor F_1^*$:\\
$$\mathcal{A}(F)=1 \Leftrightarrow \mathcal{A}(F_0)=1 ~und~\mathcal{A}(F_1)=1$$\\
$$\Leftrightarrow \mathcal{A}^*(F_0^*)=0 ~und ~\mathcal{A}^*(F_1^*) = 0$$\\
$$\Leftrightarrow \mathcal{A}^*(F^*)=0$$\\
\item F = ($F_0 \lor F_1)$ analog. $\square$\\ 
\end{itemize}
\textbf{Satz}: Es gelten folgenden Äquivalnzen sowie ihre Duale:\\
$$(A\land A ) \equiv A \text{~(Idempotenz)}$$\\
$$(A\land B) \equiv (B \land A) \text{~(Kommutativität)}$$\\
$$((A\land B) \lor C) \equiv ((A \lor C ) \land (B\lor C)) \text{~(Distributivität)}$$\\
$$(A \land (A \land B )) \equiv A \text{~(Absorption)}$$\\
$$(A \land (B \land C )) \equiv ((A\land B ) \land C) \text{~Assoziatovität)}$$\\
$$(\bot A) \equiv A$$ \\
$$(\top \land A) \equiv \top$$\\
$$(A \land \neg A) \equiv \bot \text{~"(dual zu tertum non datur")}$$\\
\textbf{Beweis}: Kommutativität, Distributivität, Absorption siehe oben. Rest analog.\\
\textbf{Def.}: Es seien F,H Formeln und A eine Variable. Dann bezeichnet die Formel F(H/A) die Formel die aus F entsteht, indem man jedes Vorkommen der Variable A durch die Formel H ersetzen. (Dabei gehen wir nicht rekursiv vor d.h. wenn H selbst die Variable A enthält, lassen wir A danach stehen).\\
\textbf{Bsp.}: F=($A\land B$), G=($B\land A$), H=($B\lor C$)\\
F(H/B)=($A\land (B\lor C$)), G(H/B)=(($B\lor C) \land A)$\\
\textbf{\underline{Lemma(Ersetzungslemma)}}: Es seien F,G,H Formeln und A eine Variable. Wenn G$\equiv$ H gilt, dann gelten auch:\\
$$F(G/A) \equiv F(H/A)(1)$$\\ und $$G(F/A) \equiv H(F/A)(2).$$\\
\textbf{\underline{Beweis}}: Zu (1) : Bei allen Belegungen $\mathcal{A}$ gilt $\mathcal{A}(G)=\mathcal{A}(H)$. Bei der rekursiven Definition von $\mathcal{A}(F)$ kann ich anstelle $\mathcal{A}(A)$ kann ich $\mathcal{A}(G)~oder~\mathcal{A}(H)$ einsetzen und erhalte beidemal das gleiche Ergebniss für alle Belegungen $\Rightarrow$ (1).\\
Zu (2): In der rekursiven Definition von $\mathcal{A}(G),\mathcal{A}(H)$ ersetze wieder $\mathcal{A}(A)~durch~\mathcal{A}(F)$, es folgt wieder die Äquivalenz (2). $\square$\\
\textbf{\underline{WARNUNG}}: Im Allgemeinen folgt aus (1) oder (2) nicht, dass G$\equiv$H.\\
\textbf{\underline{Satz}}: Es gelten die de Morgenschen Regeln:\\
$\neg(A\land B) \equiv (\neg A \lor \neg B)$\\
dual dazu: $\neg(A \lor B) \equiv (\neg A \land B )$\\
$\neg\neg\neg A \equiv A$.\\z
\textbf{\underline{Beweis}}: Wie oben, siehe auch Beweis des Dualitätslemma.$\square$\\
\textbf{\underline{Satz}}: Ein \underline{Literal} ist eine Variable A, oder $\neg $A. Ein Ausdruck der Form:\\
$\bigvee\limits_{i<m}\bigwedge\limits_{j<n_i} L_{ij}$\\
wobei $L_{ij}$ Literale sind, heißt "disjunktive Normalform". Dual dazu heißt:\\
$\bigwedge\limits_{i<m'}\bigvee\limits_{j<n'_i}L'_{ij}$\\
"konjunktive Normalform".\\
\textbf{\underline{Satz}}: Jede Formel F ist zu einer Formel in konjunktiver, bzw. disjunktiver Normalform äquivalent.\\
\textbf{\underline{WARNUNG}}: Diese Normalformen sind nicht eindeutig, z.B. sind:\\
$$A\equiv A \lor (B\land A)$$\\
beide in disjunktiver Normalform.\\
\textbf{\underline{Beweis}}: Induktiv für beide Normalformenzusammen:\\
Sei etwa F = $(F_0 \land F_1)$, dann bringe zunächst $F_0$ und $F_1$ in die gewünschte Normalform. Für die konjunktive Normalform hänge die beiden großen Konjunktionen zusammen:\\
$\bigwedge\limits_{i<m_0}\bigvee\limits_{...} L'_{ij} \land \bigwedge\limits_{i<m_1} \bigvee\limits_{...} L''{ij}$\\
$\rightarrow \bigwedge\limits_{i<m_o+m_1}\bigvee\limits_{...} L_{ij}$.\\
Für die disjunktive Normalform benutzt man das Distributivgesetz:\\
$\bigvee\limits_{i<m_0}\bigwedge\limits_{...}L'_{ij} \land \bigvee\limits_{i<_m1}\bigwedge\limits_{...} L''_{ij}$\\
$\rightarrow \bigvee\limits_{i< m_0\cdot m_1} ( \bigwedge\limits_{...} L'_{ij} \land \bigwedge\limits_{...} L''_{ij})$\\
Analog verfahre mit F=($F_0\lor F_1)$ (Beides ist analog zum Ausmultiplizieren von Polynomen in mehreeren Veränderlichen ...)\\
Um F=$\neg$ G in konjunktive Normalform zu bringen, bringe G in disjunktive Normalform und wende dann die de Morgenschen Regeln:\\
$\neg (\bigvee\limits_{i<m}\bigwedge\limits_{j<n_i}L_{ij})=\bigwedge\limits_{i<m}\bigvee\limits_{j<n_i} \neg L_{ij}$\\
anschließend beseitige doppelte Veneinungen:$\neg\neg A\equiv A$. Analog für disjunktive Normalform. $\square$

\subsection{Boole'sche Algebren}
\textbf{Definition:} Eine boole'sche Algebra $(B,0,1,\sqcap,\sqcup,^\complement)$ besteht aus einer Menge $B$, Elenenten $0,1\in B$, Verknüpfungen $\sqcap,\sqcup: B \times B \rightarrow B$ und einem Komplement $^\complement\in B: a \mapsto a^\complement$, so dass folgende Axiome für $\forall a,b,c,\in B$ gelten:\\
\begin{tabular}{cccl}
(1.1) & $a \sqcup a = a$ & $a \sqcap a = a$ & (Idempotenz)\\
(1.2) & $a \sqcup b = b \sqcup a$ & $a \sqcap b = b \sqcap a$ & (Kommutativität)\\
(1.3) & $(a \sqcap b) \sqcap c = a \sqcap (b \sqcap c)$ & $(a \sqcup b) \sqcup c = a \sqcup (b \sqcup c)$ & (Assoziaivität)\\
(1.4) & $a \sqcup (a \sqcap b) = a$ & $a \sqcap (a \sqcup b) = a$ & (Absorption)\\
(1.5) & $a \sqcap (b \sqcup c) = (a \sqcap b) \sqcup (a \sqcap c)$ & $a \sqcup (b \sqcap c) = (a \sqcup b) \sqcap (a \sqcap c)$ & (Distributivität)\\
(1.6) & $0 \sqcap a = 0$ & $1 \sqcup a = 1$ & \\
(1.7) & $a \sqcap a^\complement = 0$ & $a \sqcup a^\complement = 1$ &
\end{tabular}\medskip\\
\textbf{Bemerkungen:}
\begin{itemize}
\item Eines der Distributivitätsgesetze (1.5) ist überflüssig
\item Es gelten die de Morgan'schen Regeln
\end{itemize}
\textbf{Beispiel einer Folgerung}:\\
$a \sqcup 0 \stackrel{1.6}{=} a \sqcup (a \sqcap 0) \stackrel{1.4}{=} a$\\
Analog für $a \sqcap 1 = a$\medskip\\
\textbf{Beispiel 1:} Es sei $X$ eine Menge. Dann ist die Potenzmenge $\mathcal{P}(X)$ eine boole'sche Algebra $(\mathcal{P}(X),\varnothing,X,\cap,\cup,\setminus\cdot)$.\\ Für $A \subset X$ ist $A^\complement = X \setminus A$\medskip\\
\textbf{Beispiel 2:} Ein n-Byte sei eine Folge von $n$ Bits aus $\{0,1\}$. Dann bildet die Menge aller n-Bytes eine boole'sche Albegra mit AND, OR, NOT. \medskip\\
Diese Beispiele sind isomorph. Dabei entspricht dem n-Byte $(b_0,...,b_{n-1})$ die Teilmenge $\{i \in \{0,...,n-1\}\mid b_i = 1\} \in \mathcal{P}(\{0,...,n-1\})$\medskip\\
Es gibt für boole'sche Algebren das Prinzip der Dualität:\\
Wenn eine Gleichung gilt, dann gilt sie auch nach Vertauschen von $\sqcap$ und $\sqcup$, sowie von $0$ und $1$.\medskip\\
\textbf{Definition:} Eine Struktur $(M,\sqcap,\sqcup)$ heißt \underline{Verband}, wenn (1.1) bis (1.4) gelten.\medskip\\
\textbf{Bemerkung:} Zu jedem Verband gehört eine \underline{partielle Ordnung} $\leq$ mit:
$$a \leq b \Leftrightarrow a \sqcap b = a$$
Es gelten:
\begin{description}
\item[Reflexivität:] $a \sqcap a = a \Rightarrow a \leq a$
\item[Antisymmetrie:] Es gelte $a \leq b, b \leq a; a \sqcap b = a, b \sqcap a = b$\\ Aus (1.2) folgt $a = a \sqcap b = b$
\item[Transitivität:]Es gelte $a \leq b, b \leq c\\a \sqcap b = a, b = b \sqcap c\\a=a \sqcap b = a \sqcap (b \sqcap c) = (a \sqcap b) \sqcap c = a \sqcap c$
\end{description}
\textbf{Beispiel:} Die Potenzmenge ist ein Verband mit $A \leq B \Leftrightarrow A \cap B = A \Leftrightarrow A \subseteq B$\medskip\\
Umgekehrt sei $(M,\leq)$ eine Menge mit partieller Ordnung\medskip\\
Ein \underline{Infimum} von $a,b \in M$ ist ein Element 
$$c=\inf(a,b)\text{ mit }c \leq a, c\leq b$$
so dass
$$\forall d \in M: c \leq d \leq a, c \leq d \leq b \Rightarrow d = c$$\ \medskip\\
Ein \underline{Supremum} von $a,b \in M$ ist ein Element 
$$c=\sup(a,b)\text{ mit }a \leq c, b \leq c$$
so dass
$$\forall d \in M: a \leq d \leq c, b \leq d \leq c \Rightarrow d = c$$\ \medskip\\
Wenn in einer partiellen Ordnung $(M,\leq)$ je zwei $a,b, \in M$ ein Infimum oder Supremum haben, erhalten wir einen Verband mit $a \sqcap b = \inf(a,b)$ und $a \sqcup b = \sup(a,b)$.\medskip\\
Wenn es Elemente $0,1$ mit $0 \leq a \leq 1\ \forall a \in M$ gibt, dann gilt (1.6).\medskip\\
Zu einer boole'schen Algebra fehlen noch die Komplemente, sowie die Distributivität.\medskip\\
\textbf{Beispiel:} (Verband, aber keine boole'sche Algebra)\\
Sei $V$ ein Vektorraum. Dann bilden die Untervektorräume von $V$ einen Verband $U(V)$ mit $U \leq W \Leftrightarrow U \subseteq W$.\\Dann\\
$U \sqcap W = U \cap W\\U\sqcup W = U + W =\{u+w\mid u \in U, w \in W\}$\\
Aber (1.5) ist verletzt! Sei $V = \mathbb{R}^2$:\\
$U = \mathbb{R} \times \{0\}, W = \{0\} \times \mathbb{R}$ (x- und y-Achse)\\
Sei $X = \{(x,y) \mid x \in \mathbb{R}\}$, dann:\\
$U \sqcup W = \mathbb{R}^2\\
X \sqcap (U \sqcup W) = X\\
X \sqcap U = \{0\}\\
X \sqcap W = \{0\}\\
(X \sqcap U)\sqcup (X \sqcap W) = \{0\}$\medskip\\
\textbf{Satz:} (Stone'scher Darstellungssatz)\\
Jede boole'sche Algebra ist Unteralgebra einer Potenzmengenalgebra und jede endliche boole'sche Algebra ist isomorph zu einer Potenzmengenalgebra. \medskip\\
\textbf{Beweis} der zweiten Aussage:\\
Sei $(B,0,1,\sqcap,\sqcup,^\complement)$ eine boole'sche Algebra, so dass $B$ eine endliche Menge ist.\\
Dann ist $B$ ein Verband mit Minimum $0$.\medskip\\
Ein \underline{Atom} in $B$ ist ein Element $a \in B\setminus\{0\}$, so dass aus $0 \leq b \leq a$ stets $b=0$ oder $b=a$ folgt.\medskip\\
Sei $x$ die Menge der Atome in $B$.\\
Definiere $\Phi: B\rightarrow \mathcal{P}(X)$ durch
$$\Phi(b) = \{a \in X \mid a \leq b\} \in \mathcal{P}(X)$$
Zeige zunächst: $\Phi$ ist bijektiv:\\
Surjektivität: Sei $U \subset X$, dann betrachte
$$b=\bigsqcup_{a \in U} a = a_0\sqcup ... \sqcup a_{k-1}\text{, falls } U=\{a_0,..,a_{k-1}\}$$
Sei $a \in X$, so gilt
$$a \leq b \Leftrightarrow a \sqcap(a_0 \sqcup...\sqcup a_{k-1}) = a = (a \sqcap a_0) \sqcup ... \sqcup (a \sqcap a_{k-1})$$
Da $a \sqcap a_i \leq a, a \sqcap a_i \leq a_i $ und $a,a_i$ Atome gilt entweder
$$a \sqcap a_i = 0 \Leftrightarrow a \neq a_i$$
oder
$$a = a \sqcap a_i = a_i$$
ist der obige Ausdruck entweder $0$ (falls $a \notin U$) oder $a = a_i \in U$\\
$\Rightarrow \Phi(b) = U$\medskip\\
Injektivität:\\
Es gelte $\Phi(b) = \Phi(a)$,\\ 
dann gilt $\Phi(b \sqcap c) = \Phi(c)$, denn\medskip\\
Sei $a \leq b, a \leq c, a$ Atom:\\
$a = a \sqcap b = a \sqcap c = (a \sqcap b) \sqcap c = a \sqcap (b \sqcap c)$,\\
also o.B.d.A. $b \leq c$.\smallskip\\
Betrachte $c \sqcap b^\complement$:\\
Falls $b \leq c$ und $c \sqcap b^\complement = 0$ folgt $b=c$,\\
Falls $c \sqcap b^\complement \neq 0$, finden wir $a \in X$\\
mit $a \leq c \sqcap b^\complement \Rightarrow a \in \Phi(c)\setminus \Phi(b)\ \ \lightning$\medskip\\ % Blitz aus 'stmaryrd'
Dazu nutzen wir aus, dass $B$ endlich ist:\\
jede Kette $0 < a < ... < c \sqcap b^\complement$ hat höchstens Länge $\#B$ und für längstmögliche Kette ist $a$ ein Atom.\medskip\\
$\Rightarrow \Phi$ ist bijektiv\medskip\\
Noch zu zeigen: $\Phi(0)= \varnothing, \Phi(1) = X$, usw.$\hfill\square$\medskip\\
\textbf{Beispiel:} Lindenbaum-Algebra\\
Die Lindenbaum-Algebra in $n$ Variablen $A_0,...,A_{n-1}$ ist die Menge der Äquivalnenzklassen aussagenlogischer Formeln in den Variablen $A_0,...,A_{n-1}$ bezüglich der Äquivalenzen(?) aus 1.2.\smallskip\\
Schreibe Elemente als $F/\equiv$ (z.B. $(\lnot A_0 \lor A_1)/\equiv$) also
$$L A_n = \{F/\equiv \mid F\text{ ist aussagenlogische Formel in }A_0,...,A_{n-1}\}$$
Wir setzen
\begin{itemize}
\item $0 = \bot/\equiv = (A_0 \land \lnot A_0)/\equiv$
\item $1 = \top /\equiv = (A_u \lor \lnot A_0)/\equiv$
\item $(F/\equiv)\sqcup(G/\equiv) = (F \lor G)/\equiv$
\item $(F/\equiv) \sqcap (G/\equiv) = (F \land G)/\equiv$
\item $(F/\equiv)^\complement = \lnot F/\equiv$
\end{itemize}
Aus den elementaren Äquivalenzen (Abschnitt 1.2) und dem Ersetzungslemma folgt, dass aööe Axiome einer boole'schen Algebra gelten.\medskip\\
\textbf{Warnung}: $L A_n$ ist nicht isomorph zu $\mathcal{P}(\{0,...,n-1\})$
\end{document}
