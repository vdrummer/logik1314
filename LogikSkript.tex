\documentclass[a4paper]{scrartcl}
\usepackage[utf8]{inputenc}
\usepackage[ngerman]{babel}
\usepackage{textcomp}
\usepackage{amsmath}
\usepackage{stmaryrd}
\usepackage{ulem}
\usepackage[usenames,dvipsnames,svgnames,table]{xcolor}

% zus�tzliche mathematische Symbole, AMS=American Mathematical Society 
\usepackage{amssymb}

% f�rs Einbinden von Graphiken
\usepackage{graphicx}

% f�r Namen etc. in Kopf- oder Fu�zeile
\usepackage{fancyhdr}

% Selbst definierte Farben
\definecolor{light-gray}{gray}{0.95}


%für hyperlinks auf die sections
\usepackage[colorlinks,pdfpagelabels,pdfstartview = FitH,bookmarksopen = true,bookmarksnumbered = true,linkcolor = black,plainpages = false,hypertexnames = false,citecolor = darkblue] {hyperref}

% erlaubt benutzerdefinierte Kopfzeilen 
\pagestyle{fancy}

% Definition der Kopfzeile
\lhead{
\begin{tabular}{ll}
Logik für Studierende der Informatik
\end{tabular}
}
\chead{}
\rhead{\today{}}
\lfoot{}
\cfoot{Seite \thepage}
\rfoot{} 
\parindent 0pt

% Definition neuer Befehle
\newcommand{\mfa}{\mathfrak{A}}  % Shortcut für A in Frakturschrift

\begin{document}
\tableofcontents
\newpage

% KOMMENTARE -------------------------------------------
% Ilia ( 12.11.13 ): 
% Verbesseerungsvorschlag 1: Im Header das aktuelle Kapitel anzeigen lassen, für einfachere Navigation
% Formatierung verbesseren, auch vom Quellcode. Wir haben erst ein paar Seiten und es ist jetzt schon schwer sich zurecht zu 
% finden. Nutzt beim Texen Kommentare um Blöcke/Themen/Unterthemen zu trennen. 
% Grade wenn man Abschnitte kopieren will, weil Formeln mehrfach vorkommen, ists echt ein Krampf.
%
% TO DO/ TO FIX
% - Abstände nach Absätzen/ Auflistungen zu groß? 
% - Einheitliche Darstellung von Abschnittsüberschriften(Beweis/Bemerkung/Definition)
%   Vorschlag: Fett ohne unterstrich
% - Definitionen in einer Box darstellen? So kann man schneller von Sachen finden. 
%   Im Sinne von gängigen Mathe-Büchern, wo für Definitionen und Lemma Boxen verwendet werden.
%   Oben findet sich ein 'definecolor' = hell-grau und kann mit \colorbox verwendet werden
%   Hier ne Doku: http://en.wikibooks.org/wiki/LaTeX/Colors

% HIER BEGINNT DAS SKRIPT ----------------------------------------



\section{Aussagenlogik}

\subsection{Grundbegriffe}
\textbf{Defintion:} (Syntax) Eine aussagenlogische Formel ist eine Zeichenkette, die sich aus den Variablen $A_0, A_1,$ ... und den Symbolen $(, ), \wedge, \vee, \neg$ zusammensetzt und folgende Regeln einhält:
\begin{itemize}
\item Variablen sind Formeln 
\item Wenn $F_1$ und $F_2$ Formeln sind, dann ist $(F_1 \wedge F_2)$ eine Formel (\textit{Konjunktion})
\item Wenn $F_1$ und $F_2$ Formeln sind, dann ist auch $(F_1 \vee F_2)$ eine Formel (\textit{Disjunktion})
\item Wenn F eine Formel ist, ist auch $\neg F$ eine Formel (\textit{Negation})
\end{itemize}
Alle Formeln enstehen auf diese Weise.\\
\textbf{Beispiel:} $(\neg A_0 \wedge A_1)$\\
\textbf{Lemma:} (Eindeutige Lesbarkeit) Jede Formel ist
\begin{itemize}
\item eine Variable
\item eine Konjunktion $(F_1 \wedge F_2)$
\item eine Disjunktion $(F_1 \vee F_2)$
\item eine Negation $\neg F$
\end{itemize}
Es tritt immer genau einer dieser Fälle ein, und ggf sind $F_1$ und $F_2$, bzw. $F$ eindeutig bestimmt.\\
\textbf{Beispiel:} Wir bezeichnen das erste Zeichen der gegebenen Formel. Es treten die Fälle auf:
\begin{itemize}
\item es ist eine Variable $A_i$. Dann ist $A_i$ nach Definition die ganze Formel.
\item es ist das Zeichen ''$\neg$'', dann ist nach Definition unserer Formel $\neg F$. $F$ ist also der Rest unserer Zeichenkette.
\item es ist ''$($''. Nach Definition ist unsere Formel vom Typ $(F_1 \wedge F_2)$ / $(F_1 \vee F_2)$.
\end{itemize}
Wir nehmen an, dass wir die Formel außerdem auch als  $(F'_1 \wedge F'_2)$ / $(F'_1 \vee F'_2)$ lesen können. Dann ist entweder $F'_1$ Anfangsstück von $F_1$ oder umgekehrt. Nach dem folgenden Hilfssatz gilt $F_1 = F'_1$ und das unmittelbar folgende Zeichen muss ''$\wedge$'' oder ''$\vee$'' sein und legt den Typ der Formel fest. Wenn man eine Formel nach obigem Lemma zerlegt hat und die Formel keine Variable war, dann wendet man das Lemma anschie{\''ss}end weiter auf $F_1$ und $F_2$ bzw. $F$ an bis die Formel komplett zerlegt wurde.\\
\textbf{Hilfssatz:} Kein echtes Anfangsstück einer Formel ist selbst eine Formel. Mit ''\textit{echtem Anfangsstück}'' meinen wir ein Anfangsstück, das nicht die ganze Formel ist.\\
\textbf{Beispiel:} $(F_1$ ist Anfangsstück von $(F_1 \wedge F_2)$.\\
\textbf{Beweis:} Durch Induktion über die Länge der Formel. Sei $F$ Formel der Länge 1, dann hat $F$ das echte Anfangsstück '' '' (\textit{leere Formel}), aber das ist keine Formel.\\ 

\newpage

Wir nehmen jetzt an, dass der Hilfssatz für alle kürzeren Formeln bewiesen werden kann. Wir wenden eine Fallunterscheidung nach dem ersten Buchstaben an.
\begin{itemize}
\item Variable $A_1$: Dann hat $F$ die Länge 1.
\item ''$\neg$'': Nach Definition gilt: $F = \neg F_1$. Sei $F'$ echtes Anfangsstück von $F_1$, dann folgt $F' = \neg F_1$, und $F'_1$ ist echtes Anfangsstück von $F_1$. Nach Induktionsvoraussetzung ist aber kein echtes Anfangsstück von $F_1$ eine Formel. Also kann $F'$ keine Formel sein.
\item ''$($'': Wie im obigen Beweis gilt dann $F = (F_1 \wedge F_2)$ / $(F_1 \vee F_2)$. Sei $F'$ echtes Anfangsstück von $F$, dann folgt $F' = (F_1 \wedge F_2)$ / $(F_1 \vee F_2)$. Da $F_1$ und $F_2$ kürzer als $F$ sind und $F_1$ Anfangsstück von $F'_1$ oder $F'_1$ Anfangsstück von $F_1$ ist, muss $F_1 = F'_1$ gelten. Dann muss $F'_2)$ ein Anfangsstück von $F_2)$ sein, also $F'_2$ ein Anfangsstück von $F_2$. Nach Induktionsvoraussetzung kann $F'_2$ kein echtes Anfangsstück gewesen sein.
\end{itemize}
Damit wäre der Hauptsatz bewiesen. $\square$\\
\textbf{Definition:} Es sei $\mathcal{D}$ eine Menge von Variablen, dann ist eine Belegung eine Abbildung $\mathcal{A}$:
\begin{center}
$\mathcal{A}: \mathcal{D} \rightarrow \{0, 1\}$
\end{center}
Dabei stehen 0, 1 für die Wahrheitswerte ''\textit{falsch}'' oder ''\textit{wahr}''. Sei also $A_3 \in \mathcal{D}$ und $\mathcal{A}(A_3) = 1$, dann hat $A_3$ unter der Belegung $\mathcal{A}$ den Wahrheitswert 1 (''\textit{wahr}''). Wir definieren Verknüpfungen von Wahrheitswerten:

\[ w_1 \wedge w_2 = \left\{
  \begin{array}{l l}
    1 & \quad \text{$w_1 = 1$ und $w_2 = 1$}\\
    0 & \quad \text{$w_1 = 0$ oder $w_2 = 0$}
  \end{array} \right.\]
\[ w_1 \vee w_2 = \left\{
  \begin{array}{l l}
    1 & \quad \text{$w_1 = 1$ oder $w_2 = 1$}\\
    0 & \quad \text{$w_1 = 0$ und $w_2 = 0$}
  \end{array} \right.\]
\[ \neg w_1 = \left\{
  \begin{array}{l l}
    1 & \quad \text{$w = 0$}\\
    0 & \quad \text{$w = 1$}
  \end{array} \right.\]
\textbf{Definition:} (Semantik) Sei $\mathcal{A}$ eine Belegung der Variablen einer Formel $F$, dann hat $F$ den Wahrheitswert $\mathcal{A}(F)$ von $F$:
\begin{itemize}
\item $\mathcal{A}(A_1)$, falls $F = A_1$ Variable ist
\item $\mathcal{A}(F_1) \wedge \mathcal{A}(F_2)$, falls $F = (F_1 \wedge F_2)$
\item $\mathcal{A}(F_1) \vee \mathcal{A}(F_2)$, falls $F = (F_1 \vee F_2)$
\item $\neg \mathcal{A}(F')$, falls $F = \neg F'$
\end{itemize}

\newpage

Wegen des obigen Lemma ist $\mathcal{A}(F)$ dadurch wohldefiniert. Wir verwenden folgende Abkürzungen:
\begin{itemize}
\item $(F_1 \rightarrow F_2)$ für $(\neg F_1 \vee F_2)$
\item $(F_1 \Leftrightarrow F_2)$ für $((F_1 \rightarrow F_2) \wedge (F_2 \rightarrow F_1))$
\item $\bigwedge_{i < n}{F_i}$ für ($F_1 \wedge ... (F{n-2} \wedge F{n-1}) ... )$
\item $\bigvee_{i < n}{F_i}$ für ($F_1 \vee ... (F{n-2} \vee F{n-1}) ... )$
\item $\top$ (''\textit{wahr}'') für $(\neg A_0 \vee A_0)$
\item $\bot$ (''\textit{falsch}'') für $(\neg A_0 \wedge A_0)$
\end{itemize}
\textbf{Konventionen:} Wir können $\top$, $\bot$ auch als ''\textit{nicht zusammengesetzte}'' Formeln betrachten
\begin{itemize}
\item $\bigwedge\limits_{i < 0}{F_i} = \top$
\item $\bigvee\limits_{i < 0}{F_i} = \bot$
\end{itemize}
Außerdem können wir $(F_1 \vee F_2)$ als Abkürzungen für $\neg(\neg F_1 \wedge \neg F_2)$ auffassen. Gelegentlich lassen wir Klammern weg, wenn keine Missverständnisse auftreten. Unsere Abkürzungen haben genau wie alle Formeln Wahrheitswerte unter vorgegebenen Bedingungen.\\
\begin{center}
$\mathcal{A}(\top) = 1, \mathcal{\bot} = 0,$ usw.
\end{center}
Sei $\mathcal(A)$ eine Belegung der Variablen von $F$. Wenn $\mathcal{A}(F) = 1$, sagen wir, ''\textit{$\mathcal{A}$ erfüllt $F$}'', ''\textit{$\mathcal{A}$ ist Modell von $F$}'', ''\textit{$\mathcal{A} \models F$}''.\\
\textbf{Definition:} Eine Formel heißt ''\textit{allgemeingültig}'' oder ''\textit{Tautologie}'', wenn $\mathcal{A}(F) = 1$ für alle möglichen Belegungen gilt. Eine Formel $F$ heißt ''\textit{erfüllbar}'', wenn es eine Belegung $\mathcal{A}$ mit $\mathcal{A}(F) = 1$ gibt.\\
\textbf{Beispiel:}
\begin{itemize}
\item $\top = (\neg A_0 \vee A_0)$ ist allgemeingültig, $A_0$ ist erfüllbar.
\item $\bot = (\neg A_0 \wedge A_0)$ ist nicht erfüllbar. $A_0$ ist nicht allgemeingültig.
\end{itemize}
\textbf{Lemma:} 
\begin{itemize}
\item $F$ ist genau dann eine Tautologie, wenn $\neg F$ nicht erfüllbar ist.
\item $F$ ist erfüllbar genau dann, wenn $\neg F$ keine Tautologie ist.
\end{itemize}
\subsection{Äquivalenz von Formeln und Normalformen}
\textbf{Definition}: Zwei Formeln F,G in den gleichen Variablen heißen \underline{äquivalent}, kurz $F\equiv G$, wenn $\mathcal{A}(F)=\mathcal{A}(G)$ für alle Belegungen der Variablen.\\ \\
\textbf{Beispiel}: F = A, G = A$\land$(A$\lor$B)\\ \\
\textbf{Bemerkung}: \begin{itemize}
\item F$\equiv G$ genau dann, wenn $F\leftrightarrow G$ allgemeingültig ist.
\item F allgemeingültig (kurz $\models$F) genau dann, wenn $F\equiv \top$
\item F erfüllt (kurz $\nvDash \neg $ F) genau dann, wenn F$\not\equiv \bot$
\item $"\equiv "$ ist eine  Äquivalenzrelation. [d.h. F$\equiv$F für alle Formeln F\\
F$\equiv G \Rightarrow G \equiv F$ für alle Formel n F, G\\
(F$\equiv G $ und $G\equiv H)\Rightarrow F \equiv H$ für alle Formeln F, G, H\\
Das bedeutet: Die Menge aller Formeln zerfällt in Äquivalnzklassen [F] mit G$\in$[F]$\Leftrightarrow F\equiv G$ ] \\
\end{itemize} 
\textbf{Bemerkung}: Methoden um Äquivalenz von Formeln zu zeigen:\\
\begin{itemize}
\item Fallunterscheidungen für die Belegungen (siehe Bsp, mündlich)
\item Alle Belegungen einsetzen und vergleichen: (A$\land B ) \equiv (B\land A)$ \\
\begin{tabular}{l|l|l}
$_A\backslash ^B$ & 0 & 1\\ \hline 0 & 0 & 0\\ \hline 1 & 0 & 1\\\end{tabular} ~~~~~~~~~~~~~~~~~
\begin{tabular}{l|l|l}
$_A\backslash ^B$ & 0 & 1 \\ \hline 0 & 0 & 0 \\ \hline 1 & 0 & 1 \end{tabular}
\item Venn-Diagramme: Für jede Variable A male eine $"$Menge$"$ $M_A$. Punkte in $M_A$ entsprechen Belegungen mit $\mathcal{A}(A)=1$, Punkte außerhalb:$\mathcal{A}(A)=0$\\ \\
\textbf{Bsp}: (($A\land B)\lor C ) \equiv ((A\lor C ) \land (B \lor C))$\\ \\
\textbf{Hier noch Grafik einfügen !!!}\\ 
\end{itemize}
\textbf{Definition} Sei F eine Formel, aufgebaut aus Variablen, den Junktoren $\neg, \land, \lor,$ sowie $\top,\bot$. Die zu F \underline{duale} Formel F$^*$ entsteht aus F durch Vertauschen von $\land, \lor$ sowie von $\top, \bot$. \\ \\
\textbf{Lemma}:F$\equiv$G genau dann, wenn F$^* \equiv$G\\ \\
\textbf{Beweis}: Es sei $\mathcal{A}$ eine Belegung . Vertausche in $\mathcal{A}$ die Werte 0,1 und erhalte eine neue Belegung $\mathcal{A}^*$ \\ \\
\textbf{Behauptung}: es gilt $\mathcal{A}(F)=1$ genau dann, wenn $\mathcal{A}^*(F^*)=0$.\\ \\
\textbf{Begründung}: Durch Induktion über Länge der Formel mit dem Lemma über eindeutige Lesbarkeit.\\ \\
$\begin{aligned}
&F = A\Rightarrow F^*=A\\
&\mathcal{A}(F) = \mathcal{A}(A)=1 \Leftrightarrow \mathcal{A}^*(A)=0=\mathcal{A}^*(F^*))\\
&F=\top \Rightarrow F^*=\bot\\
&\mathcal{A}(\top) = 1\text{ gilt immer genau } A^*(\bot)=0\\ 
&\text{Analog für }F=\bot\\
\end{aligned}$
Induktionsschritt:\\
\begin{itemize}
\item F = $\neg$ G : \\ \\
$\begin{aligned}
&\mathcal{A}(F) &= 1 &\Leftrightarrow \mathcal{A}(G) &= 0\\
\overset{IV}{\Leftrightarrow} &\mathcal{A}^*(G^*) &= 1 &\Leftrightarrow \mathcal{A}^*(F^*) &= 0 \end{aligned}$
\item F = ($F_0 \land F_1) , F^*=(F_0^*\lor F_1^*)$:\\ \\
$\begin{aligned}
&\mathcal{A}(F) &&= 1 \Leftrightarrow \mathcal{A}(F_0)=1 ~und~\mathcal{A}(F_1)=1\\
\Leftrightarrow &\mathcal{A}^*(F_0^*) &&= 0 ~und ~\mathcal{A}^*(F_1^*) = 0\\
\Leftrightarrow &\mathcal{A}^*(F^*) &&= 0 \end{aligned}$\\
\item F = ($F_0 \lor F_1)$ analog. $\square$\\ 
\end{itemize}
\textbf{Satz}: Es gelten folgenden Äquivalnzen sowie ihre Duale:\\ \\
\begin{enumerate}
\item $(A\land A ) \equiv A \text{~(Idempotenz)}$
\item $(A\land B) \equiv (B \land A) \text{~(Kommutativität)}$
\item $((A\land B) \lor C) \equiv ((A \lor C ) \land (B\lor C)) \text{~(Distributivität)}$
\item $(A \land (A \land B )) \equiv A \text{~(Absorption)}$
\item $(A \land (B \land C )) \equiv ((A\land B ) \land C) \text{~Assoziatovität)}$
\item $(\bot \lor A) \equiv A$ 
\item $(\top \land A) \equiv \top$
\item $(A \land \neg A) \equiv \bot \text{~"(dual zu tertum non datur")}$
\end{enumerate}
\textbf{Beweis}: Kommutativität, Distributivität, Absorption siehe oben. Rest analog.\\ \\
\textbf{Definition}: Es seien F,H Formeln und A eine Variable. Dann bezeichnet die Formel F(H/A) die Formel die aus F entsteht, indem man jedes Vorkommen der Variable A durch die Formel H ersetzen. (Dabei gehen wir nicht rekursiv vor d.h. wenn H selbst die Variable A enthält, lassen wir A danach stehen).\\ \\
\textbf{Beispiel}: F=($A\land B$), G=($B\land A$), H=($B\lor C$)\\
F(H/B)=($A\land (B\lor C$)), G(H/B)=(($B\lor C) \land A)$\\ \\
\textbf{Lemma(Ersetzungslemma)}: Es seien F,G,H Formeln und A eine Variable. Wenn G$\equiv$ H gilt, dann gelten auch:\\
$$F(G/A) \equiv F(H/A)~~(1)$$\\ und $$G(F/A) \equiv H(F/A)~~(2).$$\\ \\
\textbf{Beweis}: Zu (1) : Bei allen Belegungen $\mathcal{A}$ gilt $\mathcal{A}(G)=\mathcal{A}(H)$. Bei der rekursiven Definition von $\mathcal{A}(F)$ kann ich anstelle $\mathcal{A}(A)$ kann ich $\mathcal{A}(G)~oder~\mathcal{A}(H)$ einsetzen und erhalte beidemal das gleiche Ergebniss für alle Belegungen $\Rightarrow$ (1).\\
Zu (2): In der rekursiven Definition von $\mathcal{A}(G),\mathcal{A}(H)$ ersetze wieder $\mathcal{A}(A)~durch~\mathcal{A}(F)$, es folgt wieder die Äquivalenz (2). $\square$\\ \\
\textbf{\underline{WARNUNG}}: Im Allgemeinen folgt aus (1) oder (2) nicht, dass G$\equiv$H.\\ \\
\textbf{Satz}: Es gelten die de Morgenschen Regeln:\\ 
$\neg(A\land B) \equiv (\neg A \lor \neg B)$\\
dual dazu: $\neg(A \lor B) \equiv (\neg A \land \neg B )$\\
$\neg\neg\neg A \equiv A$.\\ \\
\textbf{Beweis}: Wie oben, siehe auch Beweis des Dualitätslemma.$\square$\\ \\
\textbf{Satz}: Ein \underline{Literal} ist eine Variable A, oder $\neg $A. Ein Ausdruck der Form:\\
$$\bigvee\limits_{i<m}\bigwedge\limits_{j<n_i} L_{ij}$$\\
wobei $L_{ij}$ Literale sind, heißt "disjunktive Normalform". Dual dazu heißt:\\
$$\bigwedge\limits_{i<m'}\bigvee\limits_{j<n'_i}L'_{ij}$$\\
$"$konjunktive Normalform$"$.\\ \\
\textbf{Satz}: Jede Formel F ist zu einer Formel in konjunktiver, bzw. disjunktiver Normalform äquivalent.\\ \\
\textbf{\underline{WARNUNG}}: Diese Normalformen sind nicht eindeutig, z.B. sind:\\ 
$$A\equiv A \lor (B\land A)$$\\
beide in disjunktiver Normalform.\\ \\
\textbf{Beweis}: Induktiv für beide Normalformenzusammen:\\
Sei etwa F = $(F_0 \land F_1)$, dann bringe zunächst $F_0$ und $F_1$ in die gewünschte Normalform. Für die konjunktive Normalform hänge die beiden großen Konjunktionen zusammen:\\
$$\bigwedge\limits_{i<m_0}\bigvee\limits_{...} L'_{ij} \land \bigwedge\limits_{i<m_1} \bigvee\limits_{...} L''{ij}$$\\ \\
$$\rightarrow \bigwedge\limits_{i<m_o+m_1}\bigvee\limits_{...} L_{ij}$$.\\
Für die disjunktive Normalform benutzt man das Distributivgesetz:\\
$$\bigvee\limits_{i<m_0}\bigwedge\limits_{...}L'_{ij} \land \bigvee\limits_{i<_m1}\bigwedge\limits_{...} L''_{ij}$$\\ \\
$$\rightarrow \bigvee\limits_{i< m_0\cdot m_1} ( \bigwedge\limits_{...} L'_{ij} \land \bigwedge\limits_{...} L''_{ij})$$\\
Analog verfahre mit F=($F_0\lor F_1)$ (Beides ist analog zum Ausmultiplizieren von Polynomen in mehreren Veränderlichen ...)\\
Um F=$\neg$ G in konjunktive Normalform zu bringen, bringe G in disjunktive Normalform und wende dann die de Morgenschen Regeln:\\
$$\neg (\bigvee\limits_{i<m}\bigwedge\limits_{j<n_i}L_{ij})=\bigwedge\limits_{i<m}\bigvee\limits_{j<n_i} \neg L_{ij}$$\\
anschließend beseitige doppelte Veneinungen:$\neg\neg A\equiv A$. Analog für disjunktive Normalform. $\square$

\subsection{Boole'sche Algebren}
\textbf{Definition:} Eine boole'sche Algebra $(B,0,1,\sqcap,\sqcup,^\complement)$ besteht aus einer Menge $B$, Elenenten $0,1\in B$, Verknüpfungen $\sqcap,\sqcup: B \times B \rightarrow B$ und einem Komplement $^\complement\in B: a \mapsto a^\complement$, so dass folgende Axiome für $\forall a,b,c,\in B$ gelten:\\
\begin{tabular}{cccl}
(1.1) & $a \sqcup a = a$ & $a \sqcap a = a$ & (Idempotenz)\\
(1.2) & $a \sqcup b = b \sqcup a$ & $a \sqcap b = b \sqcap a$ & (Kommutativität)\\
(1.3) & $(a \sqcap b) \sqcap c = a \sqcap (b \sqcap c)$ & $(a \sqcup b) \sqcup c = a \sqcup (b \sqcup c)$ & (Assoziaivität)\\
(1.4) & $a \sqcup (a \sqcap b) = a$ & $a \sqcap (a \sqcup b) = a$ & (Absorption)\\
(1.5) & $a \sqcap (b \sqcup c) = (a \sqcap b) \sqcup (a \sqcap c)$ & $a \sqcup (b \sqcap c) = (a \sqcup b) \sqcap (a \sqcap c)$ & (Distributivität)\\
(1.6) & $0 \sqcap a = 0$ & $1 \sqcup a = 1$ & \\
(1.7) & $a \sqcap a^\complement = 0$ & $a \sqcup a^\complement = 1$ &
\end{tabular}\medskip\\
\textbf{Bemerkungen:}
\begin{itemize}
\item Eines der Distributivitätsgesetze (1.5) ist überflüssig
\item Es gelten die de Morgan'schen Regeln
\end{itemize}
\textbf{Beispiel einer Folgerung}:\\
$a \sqcup 0 \stackrel{1.6}{=} a \sqcup (a \sqcap 0) \stackrel{1.4}{=} a$\\
Analog für $a \sqcap 1 = a$\medskip\\
\textbf{Beispiel 1:} Es sei $X$ eine Menge. Dann ist die Potenzmenge $\mathcal{P}(X)$ eine boole'sche Algebra $(\mathcal{P}(X),\varnothing,X,\cap,\cup,\setminus\cdot)$.\\ Für $A \subset X$ ist $A^\complement = X \setminus A$\medskip\\
\textbf{Beispiel 2:} Ein n-Byte sei eine Folge von $n$ Bits aus $\{0,1\}$. Dann bildet die Menge aller n-Bytes eine boole'sche Albegra mit AND, OR, NOT. \medskip\\
Diese Beispiele sind isomorph. Dabei entspricht dem n-Byte $(b_0,...,b_{n-1})$ die Teilmenge $\{i \in \{0,...,n-1\}\mid b_i = 1\} \in \mathcal{P}(\{0,...,n-1\})$\medskip\\
Es gibt für boole'sche Algebren das Prinzip der Dualität:\\
Wenn eine Gleichung gilt, dann gilt sie auch nach Vertauschen von $\sqcap$ und $\sqcup$, sowie von $0$ und $1$.\medskip\\
\textbf{Definition:} Eine Struktur $(M,\sqcap,\sqcup)$ heißt \underline{Verband}, wenn (1.1) bis (1.4) gelten.\medskip\\
\textbf{Bemerkung:} Zu jedem Verband gehört eine \underline{partielle Ordnung} $\leq$ mit:
$$a \leq b \Leftrightarrow a \sqcap b = a$$
Es gelten:
\begin{description}
\item[Reflexivität:] $a \sqcap a = a \Rightarrow a \leq a$
\item[Antisymmetrie:] Es gelte $a \leq b, b \leq a; a \sqcap b = a, b \sqcap a = b$\\ Aus (1.2) folgt $a = a \sqcap b = b$
\item[Transitivität:]Es gelte $a \leq b, b \leq c\\a \sqcap b = a, b = b \sqcap c\\a=a \sqcap b = a \sqcap (b \sqcap c) = (a \sqcap b) \sqcap c = a \sqcap c$
\end{description}
\textbf{Beispiel:} Die Potenzmenge ist ein Verband mit $A \leq B \Leftrightarrow A \cap B = A \Leftrightarrow A \subseteq B$\medskip\\
Umgekehrt sei $(M,\leq)$ eine Menge mit partieller Ordnung\medskip\\
Ein \underline{Infimum} von $a,b \in M$ ist ein Element 
$$c=\inf(a,b)\text{ mit }c \leq a, c\leq b$$
so dass
$$\forall d \in M: c \leq d \leq a, c \leq d \leq b \Rightarrow d = c$$\ \medskip\\
Ein \underline{Supremum} von $a,b \in M$ ist ein Element 
$$c=\sup(a,b)\text{ mit }a \leq c, b \leq c$$
so dass
$$\forall d \in M: a \leq d \leq c, b \leq d \leq c \Rightarrow d = c$$\ \medskip\\
Wenn in einer partiellen Ordnung $(M,\leq)$ je zwei $a,b, \in M$ ein Infimum oder Supremum haben, erhalten wir einen Verband mit $a \sqcap b = \inf(a,b)$ und $a \sqcup b = \sup(a,b)$.\medskip\\
Wenn es Elemente $0,1$ mit $0 \leq a \leq 1\ \forall a \in M$ gibt, dann gilt (1.6).\medskip\\
Zu einer boole'schen Algebra fehlen noch die Komplemente, sowie die Distributivität.\medskip\\
\textbf{Beispiel:} (Verband, aber keine boole'sche Algebra)\\
Sei $V$ ein Vektorraum. Dann bilden die Untervektorräume von $V$ einen Verband $U(V)$ mit $U \leq W \Leftrightarrow U \subseteq W$.\\Dann\\
$U \sqcap W = U \cap W\\U\sqcup W = U + W =\{u+w\mid u \in U, w \in W\}$\\
Aber (1.5) ist verletzt! Sei $V = \mathbb{R}^2$:\\
$U = \mathbb{R} \times \{0\}, W = \{0\} \times \mathbb{R}$ (x- und y-Achse)\\
Sei $X = \{(x,y) \mid x \in \mathbb{R}\}$, dann:\\
$U \sqcup W = \mathbb{R}^2\\
X \sqcap (U \sqcup W) = X\\
X \sqcap U = \{0\}\\
X \sqcap W = \{0\}\\
(X \sqcap U)\sqcup (X \sqcap W) = \{0\}$\medskip\\
\textbf{Satz:} (Stone'scher Darstellungssatz)\\
Jede boole'sche Algebra ist Unteralgebra einer Potenzmengenalgebra und jede endliche boole'sche Algebra ist isomorph zu einer Potenzmengenalgebra. \medskip\\
\textbf{Beweis} der zweiten Aussage:\\
Sei $(B,0,1,\sqcap,\sqcup,^\complement)$ eine boole'sche Algebra, so dass $B$ eine endliche Menge ist.\\
Dann ist $B$ ein Verband mit Minimum $0$.\medskip\\
Ein \underline{Atom} in $B$ ist ein Element $a \in B\setminus\{0\}$, so dass aus $0 \leq b \leq a$ stets $b=0$ oder $b=a$ folgt.\medskip\\
Sei $x$ die Menge der Atome in $B$.\\
Definiere $\Phi: B\rightarrow \mathcal{P}(X)$ durch
$$\Phi(b) = \{a \in X \mid a \leq b\} \in \mathcal{P}(X)$$
Zeige zunächst: $\Phi$ ist bijektiv:\\
Surjektivität: Sei $U \subset X$, dann betrachte
$$b=\bigsqcup_{a \in U} a = a_0\sqcup ... \sqcup a_{k-1}\text{, falls } U=\{a_0,..,a_{k-1}\}$$
Sei $a \in X$, so gilt
$$a \leq b \Leftrightarrow a \sqcap(a_0 \sqcup...\sqcup a_{k-1}) = a = (a \sqcap a_0) \sqcup ... \sqcup (a \sqcap a_{k-1})$$
Da $a \sqcap a_i \leq a, a \sqcap a_i \leq a_i $ und $a,a_i$ Atome gilt entweder
$$a \sqcap a_i = 0 \Leftrightarrow a \neq a_i$$
oder
$$a = a \sqcap a_i = a_i$$
ist der obige Ausdruck entweder $0$ (falls $a \notin U$) oder $a = a_i \in U$\\
$\Rightarrow \Phi(b) = U$\medskip\\
Injektivität:\\
Es gelte $\Phi(b) = \Phi(a)$,\\ 
dann gilt $\Phi(b \sqcap c) = \Phi(c)$, denn\medskip\\
Sei $a \leq b, a \leq c, a$ Atom:\\
$a = a \sqcap b = a \sqcap c = (a \sqcap b) \sqcap c = a \sqcap (b \sqcap c)$,\\
also o.B.d.A. $b \leq c$.\smallskip\\
Betrachte $c \sqcap b^\complement$:\\
Falls $b \leq c$ und $c \sqcap b^\complement = 0$ folgt $b=c$,\\
Falls $c \sqcap b^\complement \neq 0$, finden wir $a \in X$\\
mit $a \leq c \sqcap b^\complement \Rightarrow a \in \Phi(c)\setminus \Phi(b)\ \ \lightning$\medskip\\ % Blitz aus 'stmaryrd'
Dazu nutzen wir aus, dass $B$ endlich ist:\\
jede Kette $0 < a < ... < c \sqcap b^\complement$ hat höchstens Länge $\#B$ und für längstmögliche Kette ist $a$ ein Atom.\medskip\\
$\Rightarrow \Phi$ ist bijektiv\medskip\\
Noch zu zeigen: $\Phi(0)= \varnothing, \Phi(1) = X$, usw.$\hfill\square$\medskip\\
\textbf{Beispiel:} Lindenbaum-Algebra\\
Die Lindenbaum-Algebra in $n$ Variablen $A_0,...,A_{n-1}$ ist die Menge der Äquivalnenzklassen aussagenlogischer Formeln in den Variablen $A_0,...,A_{n-1}$ bezüglich der Äquivalenzen(?) aus 1.2.\smallskip\\
Schreibe Elemente als $F/\equiv$ (z.B. $(\lnot A_0 \lor A_1)/\equiv$) also
$$L A_n = \{F/\equiv \mid F\text{ ist aussagenlogische Formel in }A_0,...,A_{n-1}\}$$
Wir setzen
\begin{itemize}
\item $0 = \bot/\equiv = (A_0 \land \lnot A_0)/\equiv$
\item $1 = \top /\equiv = (A_u \lor \lnot A_0)/\equiv$
\item $(F/\equiv)\sqcup(G/\equiv) = (F \lor G)/\equiv$
\item $(F/\equiv) \sqcap (G/\equiv) = (F \land G)/\equiv$
\item $(F/\equiv)^\complement = \lnot F/\equiv$
\end{itemize}
Aus den elementaren Äquivalenzen (Abschnitt 1.2) und dem Ersetzungslemma folgt, dass allee Axiome einer boole'schen Algebra gelten.\medskip\\
\textbf{Warnung}: $L A_n$ ist nicht isomorph zu $\mathcal{P}(\{0,...,n-1\})$\\\\

% Resolutionsmethode 11.11.13
\subsection{Resolutionsmethode}
\textbf{Bemerkung:} Um F durch eine Formel in KNF zu ersetzen, braucht man nur `elementare Äquivalenzen' und das Ersetzungslemma. Die 'elementaren Äquivalenzen‘ (Satz 1.2) entsprechen den  Axiomen der Boolschen Algebra (für die Lindenbaum Algebra). \\
\textbf{Bemerkung:} Um die konjunktive Normalform eindeutig zu machen, vereinbaren wir, dass in 

$$\bigwedge \limits_{i<m}\bigvee\limits_{j<n_{i}} L_{ij} $$

die Ausdrücke $\bigvee\limits_{j<n_{i}} L_{ij}$ Atome sind. \medskip\\
Das heißt $(\neg )$~$A_{0}~ \vee (\neg )$~$A_{1}~ \vee \cdots \vee (\neg )$~$A_{n-1}$  falls $A_{0}$, $\cdots A_{n-1}$ alle vorkommenden Variablen sind.\\\\\\\\

\textbf{Beispiel:}  $(a \vee b) \wedge (\neg a \vee b) \wedge (a \vee \neg b)  \wedge (\neg a \vee \neg b)$  nicht erfüllbar\medskip\\ 
\textbf{Beispiel:}  Falls \{A,B\} die Variablen sind:

$$ A \equiv (A \vee B) \wedge (A \vee B)$$
(Fallunterscheidung nach $\mathcal{A}$(B)) \medskip\\
\textbf{Beweis:} Diese Zerlegung in Atome macht die konjunktive Normalform eindeutig (bis auf Reihenfolge), aber die neue Formel hat Länge \\
$$O(2^{\# Variablen})$$
\begin{itemize}
\item Eine solche Formel F ist allgemeingültig genau dann, wenn sie alle mögliche keine Atome enthält 
\item Eine solche Formel F ist nicht erfüllbar genau dann, wenn sie alle mögliche Atome enthält  
\end{itemize}
Wir erhalten einen Algorithmus, der 'Allgemeingültigkeit' entscheidet, aber mit einer Laufzeit von $O(2^{\# Variablen})$.\\
Äquivalenz zu diesem Verfahren ist die 'Brute Force'-Methode:\\
Es gibt $2^{\# Variablen}$ viele Belegungen.
Berechne $\mathcal{A}$(F) für jede Belegung $\mathcal{A}$, um Allgemeingültigkeit (oder Erfüllbarkeit) zu entscheiden.\medskip\\
\textbf{Notation:} Schreibe eine Formel \\
$$\colorbox{light-gray}{$\bigwedge \limits_{i<m}\bigvee\limits_{j<n_{i}} L_{ij} $} $$

$$\colorbox{light-gray}{$\bigwedge \limits_{i<m}\bigvee\limits_{j<n_{i}} L_{ij}$ } $$
in KNF als Menge von Klauseln \\
$$\colorbox{light-gray}{$\bigvee\limits_{j<n_{i}} L_{ij} = \{L_{i,0}, \cdots, L_{i,n-1}\}$}$$
Eine Klausel ist also eine Menge von Literalen (Variablen oder negierte Variablen). \medskip\\
Sei also $\mathcal{C}$ = \{$C_{0}, \cdots , C_{m-1}$ \} eine Menge von Klauseln und $\mathcal{A}$ eine Belegung, dann gilt:\medskip\\
$\mathcal{A} \models \mathcal{C} \Leftrightarrow \mathcal{A} \models C_{i}$ für alle $i < m$. \medskip\\
$\mathcal{A} \models \mathcal{C}_{i} = \{L_{i,0},\cdots, L_{i,n-1}\} \Leftrightarrow$ Es gibt ein $j<n$ mit $A \models L_{ij}$ \bigskip\\\
\textbf{Definition:} Es seien
\begin{flalign*}
	\colorbox{light-gray}{$C_{1} = \{      A \}\cup P $} \\
	\colorbox{light-gray}{$C_{2} = \{ \neg A \}\cup Q$}
\end{flalign*}
wobei die Klauseln $P$, $Q$ weder $A$ noch $\neg A$ enthalten. \bigskip\\
Dann heißt \colorbox{light-gray}{$C = P \cup Q $} eine Resultante von $C_{1}$, $C_{2}$.\\ 
Eine Menge $\mathcal{C}$ von Klauseln heißt unter Resultanten abgeschloßen, wenn sie mit je zwei Klauseln $C_{1}$, $C_{2}$ auch alle ihre Resultanten enthält. \bigskip\\


\textbf{Lemma:} Es sei $\mathcal{A}$ eine Belegung der Variablen von $C_{1}$, $C_{2}$.\\
Dann gilt für die Resultante $C$, dass 
\colorbox{light-gray}{$\mathcal{A} \models \{ C_{1},C_{2} \} \Leftrightarrow \mathcal{A} \models \{ C_{1},C_{2}, C \}$}.\medskip\\

\textbf{Beweis:} Fallunterscheidung nach $\mathcal{A}(A)$
\begin{enumerate}
      \item $\mathcal{A}(A)=0$: Dann gilt \colorbox{light-gray}{$\mathcal{A}\models C_{1} \Leftrightarrow A \models P$, mit $C_{1} = \{A\} \cup P$}
      \item[] Insbesondere folgt $A \models P \cup Q = C$
      \item $\mathcal{A}(A)= 1$: Dann gilt \colorbox{light-gray}{$\mathcal{A} \models C_{2} = \{ \neg A \} \cup Q \Leftrightarrow A \models Q$, also folgt $A \models P \cup Q = C$  }
  \end{enumerate}

Daraus folgt: \\ 
\colorbox{light-gray}{$\mathcal{A} \models \{C_{1},C_{2}\} \Rightarrow \mathcal{A} \models \{C_{1},C_{2}, C\} $} \medskip\\
Umgekehrt ist klar: \\ 
\colorbox{light-gray}{$\mathcal{A} \models \{C_{1},C_{2}, C\} \Rightarrow \mathcal{A} \models \{C_{1},C_{2}\} $} \\

\subsubsection{Satz} Es sei $\mathcal{C}$ eine Menge von Klauseln die unter Resultanten abgeschloßen ist.
Dann ist $\mathcal{C}$ erfüllbar genau dann, wenn $\emptyset \notin \mathcal{C}$.\\

\textbf{Beweis:} Klausel $\emptyset \notin \mathcal{C} \Rightarrow A \models \mathcal{C}$ für alle $\mathcal{A}$ dann $\mathcal{A} \neq \emptyset$.\\
Für '$\Leftarrow$' sei $\mathcal{C}$ unter Resultanten abgeschlossen. \bigskip\\
Induktion über Anzahl der Variablen:\\
Es sei $A = A_{n-1}$ die 'größte' Variable (größter Index), die in $\mathcal{C}$ vorkommt.\\
Wäre $\{A\}$, $\{\neg A\} \in \mathcal{C}$, dann wäre $\emptyset$ als Resultante in $\mathcal{C}$ enthalten.\\
Also dürfen wir annehmen, dass $\{ \neg A \} \notin \mathcal{C}$ (ohne Einschränkung).\\
Wir betrachten jetzt nurnoch Belegungen $\mathcal{A}$ mit $\mathcal{A}=1$.\\
Dann bearbeite $\mathcal{C}$ wie folgt:
\begin{itemize}
\item Falls $C \in \mathcal{C}$ weder $A$ noch $\neg A$ enthält, behalte $C \in \mathcal{C'}$
\item Falls $C \in \mathcal{C}$ die Variable $A$ enthält, gilt $A \models C$. Dann streiche $C$ in $\mathcal{C}$
\item Falls $C \in \mathcal{C}$ die Variable $\neg A$ enthält, sei $C \models \{ \neg A \} \cup Q$, dann streiche $\neg A$, somit sei $Q \in \mathcal{C'}$
\item (Falls $C \in \mathcal{C}$ weder $A$ noch $\neg A$ enthält, wenn c bereits allgemeingültig, wir hätten also $\mathcal{C}\backslash \{C\}$ betrachten können)
\end{itemize}
So erhalten wir eine neue Klauselmenge $\mathcal{C'}$, und es gilt \colorbox{light-gray}{$\mathcal{A'} \models \mathcal{C'} \Rightarrow \mathcal{A} = \mathcal{A'} \cup \{A \rightarrow 1\} \models \mathcal{C}$} \\
Wenn also $\mathcal{C'}$ erfüllbar ist, ist $\mathcal{C}$ erst recht erfüllbar
\begin{itemize}
\item $\mathcal{C'} \notin \emptyset$ dann nach Annahme $\emptyset \notin \mathcal{C}$ und $\{\neg A \} \notin \mathcal{C}$
% ist das nächste ein C' oder ein C ?
\item $\mathcal{C'}$ ist unter Resultanten abgeschloßen.
\end{itemize} 
Dann sei B eine Variable ($B \neq A$) und \colorbox{light-gray}{$C_{1}' = \{B\} \cup P' \in \mathcal{C'}$}, \colorbox{light-gray}{$C_{2}' = \{\neg B\} \cup Q' \in \mathcal{C'}$}\bigskip\\
Dann bekommen $\mathcal{C_{2}'}$, $\mathcal{C_{2}'}$ von Klauseln der Form \medskip\\
\colorbox{light-gray}{$C_{1}= \{\neg A,B\} \cup P'$ oder $C_{1}=C_{1}'$}\\
\colorbox{light-gray}{$C_{2}= \{\neg A,\neg B\} \cup Q'$ oder $C_{2}=C_{2}'$}\medskip\\
Somit enthielt $\mathcal{C}$ die Resultante $\{\neg A\} \cup P' \cup Q'$ oder $P' \cup Q'$.\\
Damit enthält $\mathcal{C'}$ dann die Resultante $P' \cup Q'$ von $C_{1}'$ und $C_{2}'$.\\
Also ist $\mathcal{C'}$ unter Resultanten abgeschlossen mit $\emptyset \notin \mathcal{C'}$ und enthält eine Variable weniger, so dass wir die Induktionsvorraussetzung anwenden dürfen. Wenn wir keine Variablen übrig haben, gibt es nur die leere Klausel $\emptyset$, aber  $\emptyset \notin \mathcal{C}$ nach Annahme, also $\mathcal{C} = \emptyset$, und $\mathcal{C}$ steht dann \medskip\\
\colorbox{light-gray}{$\bigwedge \limits_{i<0} ? = \top $}\medskip\\
also ist $\mathcal{C}$ erfüllbar.\medskip\\
\textbf{Bemerkung:} Der Beweis liefert auch eine Belegung $\mathcal{A}$, die die ursprüngliche Klauselmenge $\mathcal{C}$ erfüllt.\\
Aber: Abschließen unter Resultanten kannn eine Laufzeit bis zu $4^{\#Variablen}$ ($3^{\#Variablen}$, falls $A_{i}$, $\neg A_{i}$ nicht simultan vorkommen) haben = nicht sehr effizient. \\

% Satz von Cook
\subsection{Der Satz von Cook}
\textbf{Ziel:} SAT = 'Saturierbarkeit', 'Erfüllbarkeit von Formeln' der Aussagenlogik ist ein NP-vollständiges Problem. \bigskip\\
\textbf{Definition:}
Es sei $\mathbb{A}$ ein Alphabet, also eine endliche Menge von Zeichen.\\
Dann bezeichnet $\mathbb{A}$* die Menge aller endlichen Wörter mit Symbolen aus $\mathbb{A}$ und $W \subset \mathbb{A}$* heißt ein Problem. \bigskip\\
\textbf{Beispiel:} Es sei \colorbox{light-gray}{$\mathbb{A} = \{ \wedge, \vee, \neg, (, ), A_{0}, \cdots ,  A_{n-1}\}$}\\
$W \subset \mathbb{A}$* sei die Menge aller Aussagenlogischen Formeln. Sei $w \in \mathbb{A}$ ein Wort der Länge $n$. Dann können sie in $O(n)$ Schritten entscheiden, ob $w \in W$.\bigskip\\\\
Betrachte dazu ein Unterprogramm 'Formel', das sich anhand des nächsten Zeichens wie folgt verhält: \\

\begin{itemize}
\item $\vee, \wedge, )$ - Fehler
\item $\neg$ - lasse das Programm 'Formel' ab dem nächsten Zeichen laufen. Wenn nicht $\vee$ oder $\wedge$ kommt - Fehler. Lasse 'Formel' ab dem Zeichen danach weiterlaufen.
\item $)$ - Überspringe und ok. Wenn $)$ kommt, überspringe und ok, ansonsten Fehler
\item Sonst - Zeichen ist Variable, überspringe sie und ok.
\end{itemize}
\textbf{Wiederholung:} Erfüllbarkeit
\begin{itemize}
\item brute force - $O(2^k)$
\item Resolutionsmethode - $O(3^k)$
\end{itemize}
k - Anzahl der Variablen \\
Auf der anderen Seite - gegeben $\mathcal{A}$ lässt sich $A \models F$ in polynomialer Zeit (in $|F|$) prüfen.\bigskip\\\\
\textbf{Ziel:} Erfüllbarkeit ist ein NP-vollständiges Problem.
\bigskip\\\\
\textbf{Bemerkung:} $F$  ist Tautologie $\Leftrightarrow$ $\neg F$ nicht erfüllbar.\\
$F \models G \Leftrightarrow F \rightarrow G$ allgemeingültig.\\
$\Leftrightarrow F \wedge \neg G$ nicht erfüllbar
\bigskip\\\\
\textbf{Erinnerung:} Turing Maschinen.
\begin{itemize}
\item $\Gamma$ - Alphabet $\Gamma \supset \mathbb{A} \cup \{\_\}$ (Leerzeichen)
\item $Q$ - Zustände, $q \in Q$, $A \subset F \subset Q$
\item $F$ - Haltezustände
\item $A$ - Akzeptable Zustände, $\delta \subset ((Q \backslash F) \times \Gamma) \times (Q \times \Gamma \times \{-1, 0, 1\})$
\item "Programm"
\end{itemize}
Dabei muss jedem $g \in Q \backslash F, a \in \Gamma$ Werte, $g \in Q, a \in \Gamma, d = +1 / -1$ existieren, sodass $(q,a, q', a', d) \in \delta$. Eine Turing-Maschine heißt deterministisch, wenn für alle $q, a$ wie oben die Werte $(q', a', d)$ eindeutig sind. Wenn wir von nicht-deterministischen Turing-Maschinen sprechen, meinen wir alle.\\
Eine Eingabe ist ein Wert der Länge $n$ aus $\mathbb{A}$, den wir auf das Band schreiben und mit "$\-$" auffüllen. Ein "Lauf" ist eine Folge von Zuständen des Bandes und der Maschine und Position des Schreiblesekopfes, sodass wenn auf der Position des Kopfes "$a$ steht und die Maschine im Zustand $"q_n"$ ist und im nächsten Schritt an der Stelle "$a_{n+1}"$ steht und die Maschine im Zustand $"q_{n+1}"$ ist und der Kopf sich um $d$ bewegt, das Tupel $(q_n, a_n, q_{n+1}, a_{n+1}, d) \in \delta$. Außerdem verändert sich der Zustand nicht, wenn $q_n \in F$.
\bigskip\\\\
\textbf{Definition:} Ein Problem $w \in \mathbb{A}$ ist von der Klasse $P$, wenn es eine deterministische Turing-Maschine gibt, die für jedes Wort $w \in \mathbb{A}^*$ der Länge $n$ in $P(n)$ Schritten entscheidet, ob $w \in W$ oder nicht. NP, wenn es eine nicht deterministische Turing-Maschine gibt, sodass für alle $w \in \mathbb{A}^*$, $n = |w|$ einen Lauf gibt, sodass für alle $w \in \mathbb{A}^*, n = |w|$ einen lauf gibt, der in $P(n)$ Schritten entscheidet ob $w \in W$. NP-vollständig, wenn es in NP liegt und jedes andere NP-Problem $w' \subset \mathbb{A}^*$ sich in polynomialer Zeit in ein äquivalentes Problem aus $w \subset \mathbb{A}^{'*}$ überführen lässt, d.h. zu jedem $w' \in \mathbb{A}^{'*}$ lässt sich in $P(|w'|)$ Schritten ein $w \in \mathbb{A}^* \Leftrightarrow w \in W$ bilden.
\bigskip\\\\
\textbf{Bemerkung:} Ein Problem ist in NP, wenn man einen Lösungsweg (z.B. eine Belegung $\mathcal{A}$ im Falle der Erfüllbarkeit) raten kann und dann innerhalb polynomialer Zeit prüfen kann ob es korrekt ist.
\bigskip\\\\
\textbf{Beispiel:} Primfaktorzerlegung ist in NP. Primzahltest in P.
\bigskip\\\\
\textbf{Bemerkung:} Vermutung: P $\neq$ NP. P $\subset$ NP gilt.\\
Falls P $\neq$ NP: P $\subset$ NP und NP-vollständig $\subset$ NP.\\
Falls P = NP: NP-vollständig $\subset$ P = NP.
\bigskip\\\\
\textbf{Satz (Cook):} Erfüllbarkeit ist NP-vollständig.
\bigskip\\\\
\textbf{Bemerkung:} Erfüllbarkeit ist in NP, denn man kann zu jeder möglichen Belegung in polynomialer Zeit prüfen, ob $\mathcal{A} \models F$.\\
Zu NP-Vollständigkeit sei $W \subset \mathbb{A}^*$ ein NP-Problem und $\delta \subset (Q \backslash F \times P) \times (Q \times \Gamma \times \{-1, 0, 1\})$ eine Turing-Maschine, die das Problem im Sinne der Definition löst. Dabei sei $P(n)$ die Laufzeit des gesuchten Laufes, dabei ist $n$ die Länge des Wortes. Wir brauchen die Variablen:
\begin{itemize}
\item $T_{ijk}$ für $- P(n) \leq i \leq P(n), j \in \Gamma, 0 \leq k \leq P(n)$, "zur Zeit $k$ steht an der Stelle $i$ des Bandes der Buchstabe $j."$ - $O(P(n)^2)$
\item $H_{ik}$ wie oben "der Schreiblesekopf steht zur Zeit $k$ an der Stelle $i"$ - $O(P(n)^2)$
\item $Q_{qk}$ wie oben, $q \in Q$ "Maschine ist zur zeit $k$ im Zustand $q"$ - $O(P(n))$
\end{itemize}
Wir geben die Formel als Menge von Klammern an, dabei gibt es verschiedene Typen:
\begin{itemize}
\item $T_{ij0}$ für alle $i$ wie oben, wobei $j$ jetzt der $i$-te Buchstabe des Eingabewortes $w \in \mathbb{A}^* sei.$ $O(P(n)$
\item $H_{00}$ Startpunkt (Initialisierung)
\item $Q_{q_{0}0}$ Startzustand (Initialisierung)
\item $\neg T_{ijk} \vee \neg T_{ij'k}$ für alle $i, k$ wie oben und alle $j \neq j' \in \Gamma$ $O(P(n)^2$ (Eindeutigkeit der Zustände)
\item $\neg Q_{qk} \vee \neg Q_{q'k}$ für alle $k$ wie oben und alle $q, q' \in Q$ (Eindeutigkeit der Zustände)
\item $\neg H_{ik} \vee \neg H_{i'k}$ für k wie oben und alle $i \neq i$ (Eindeutigkeit der Zustände)
\item $\neg T_{ijk} \vee \neg T_{ij',k+1} \vee H_{ik}$ für alle $j \neq j'$ wie oben "Band ändert sich nur am Schreiblesekopf"
\item $\neg H_{ik} \vee Q_{qk} \vee \neg T_{ijk} \vee (H_{i+0, k+1} \wedge Q_{q',k+1} \wedge T_{ij'k}) (q, j, q', j', d) \in \delta$ (Lauf)
\item "Übergangsfunktion" lässt sich umwandeln in insgesamt $O(P(n)^2)$ Klauseln (Lauf)
\item Analoge Klauseln, die besagen, dass sich der Zustand nicht ändert, wenn $Q_{qk}$ mit $q \in F$ gilt. $O(P(n)^2)$ Stück (Lauf)
\item Schluss $\bigvee\limits_{q \in A} Q_g P(n)$ beschränkte Anzahl
\end{itemize}
Eine Belegung $\mathcal{A}$ der obigen Variablen erfüllt diese Aussage genau dann, wenn sie einen Lauf der Turing-Maschine mit Eingabe $w$ beschreibt, der nach maximal $P(n)$ Schritten in einem akzeptablen Zustand terminiert. Also ist die Klauselmenge äquivalent. Zu "$w \in W$" Zeitaufwand zur Erstellung der Klauselmenge ist polynomial in $|n|$. Also:: Erfüllbarkeit ist NP-vollständig. $\square$
\bigskip\\\\
\textbf{Bemerkung:} P = NP gilt genau dann, wenn es einen polynomialen Algorithmus gibt, das die Erfüllbarkeit testet. Ein Problem $w \subset \mathbb{A}^*$ ist genau dann NP-vollständig wenn sich jede Formal $F$ in polynomialer Zeit $P(|F|)$ in ein Wort $w \in \mathbb{A}^*$ übersetzen lässt, sodass $F$ erfüllbar $\Leftrightarrow w \in W$.
\newpage
\subsection{Der Kompaktheitssatz}
\textbf{Satz:} Eine beliebige Menge $X$ von Formeln ist genau dann erfüllbar, wenn jede endliche Teilmenge erfüllbar ist.
\bigskip\\\\
\textbf{Bemerkung:} Wenn $X$ unendlich ist, muss man "unendlich lange arbeiten" um Erfüllbarkeit "von Hand" zu beweisen.
\bigskip\\\\
\textbf{Aber:} Um zu zeigen, dass $X$ \textbf{nicht} erfüllbar ist, muss man nur die richtige endliche Teilmenge "raten" und dann mit der Resolutionsmethode einen Widerspruch herleiten.\\
\underline{Wdh:} Turing Maschinen, P,NP, SAT (Erfüllbarkeit in Aussagenlogik) ist NP-Vollständig.\\
\subsection*{1.6 Der Kompaktheitsatz}
\textbf{Satz}: Eine Menge X von Formeln ist genau dann erfüllbar, wenn jede endliche Teilmenge erfüllbar ist.\\ \\
\textbf{Beweis}:Falls X nicht abzählbar ist, benötigen wir Zorn's Lemma (äquivalent zum Auswahlaxiom).\\
Sei X abzählbar, dann ist auch die Menge V der Variablen, die in X vorkommen, abzählbar:\\
$$V=\{A_n|n\in\mathbb{N}\}$$\\
Annahme: Jede endliche Teimenge von X ist erfüllbar. Wir suchen induktiv $w_i\in \{0,1\}$, für alle i$\in \mathbb{N}$, so dass für jedes n$\in\mathbb{N}$ folgendes gilt:\\
Für jede endliche Teilmenge Y$\subset$X existiert eine Belegung $\mathcal{A}$ mit $\mathcal{A}$ $\models$Y und $\mathcal{A}(A_i)=w_i$ für alle i$<n$.\\ \\
\textbf{Induktionsanfang}: n=0 $\checkmark$\\
Seien $w_0,...,w_{n-1}$ wie oben konstruiert.\\
Annahme:das gesuchte $w_i$ existiert nicht. Das heißt, für alle $w_n=0$ existiert eine endliche Teilmenge $Y_0$, so dass $\mathcal{A}\not\models Y_0$ für alle $\mathcal{A}$ mit $\mathcal{A}(A_i)=w_i$ für i$<n$, und $\mathcal{A}(A_n)=0$.\\
Analog: $Y_1 \subset X$ endlich, so das für alle $\mathcal{A}$ mit $\mathcal{A}(A_i)=w_i$ für alle $i<n$ und $\mathcal{A}(A_n)=1$ gilt:\\
$$\mathcal{A} \not\models Y_1$$\\
Dann ist $Y_0\cup Y_1$ endlich, und es existiert keine Belegung $\mathcal{A}$ mit $\mathcal{A}(A_i)=w_i$ für $i<n$ und $\mathcal{A} \models Y_0 \cup Y_1$, denn $A_n$ lässt sich nach Annahme nicht so belegen, dass die Formel aus $Y_0$ und $Y_1$ erfüllt wenden. Das ist ein Widerspruch zur Wahl von $w_0,...,w_{n-1}$. Also lassen sich die $w_n$ indukt festlegen, und wir erhalten eine     Belegung $\mathcal{A}$. Sei jetzt F$\in$X, dann sei n der maximale Index einer Variablen, die in F vorkommt (F ist endlich!). Aus der Wahl der $w_0,...,W_{n-1}$ folgt $\mathcal{A}\models \{F\}.~\square$\\ \\
\textbf{Beispiel}: "$A_{X,r}$=1$\Leftrightarrow$ Var. X hat den Wert r$\in \mathbb{R}$"\\
X heißt abzählbar genau dann, wenn es eine Folge ($x_n)~n\in\mathbb{N}$ gibt so dass:\\
$$X=\{x_n|n\in\mathbb{N}\}$$\\
$\mathbb{Q}$ abzählbar, $\mathbb{R}$ überabzählbar (=nicht abzählbar)\\ \\
\textbf{Folgerung}: Es sei X eine Menge von Formeln, und es sei G eine Formel, so dass X$\models$ G, dann existiert eine endliche Teilmenge $X_0\subset X$ und X$\models$G.\\ \\
\textbf{Beweis}: ÜBUNG!!! $\square$\\ \\
\textbf{Beispiel}: Ein Graph ist eine Menge von Punkten "Ecken" und eine Menge von Kanten (=Linien zwischen diesen Punkten).Er heißt "n-färbbar" , wenn ich jeder Ecke eine "Farbe" 0,...,n-1 zuordnen kann, so dass eine Kante zwei Ecken gleicher Farbe 
verbindet (z.B Eine pro Land auf der Landkarte, eine Kante pro gemeinsamer Grenze). Ein unendlicher Graph ist genau dann n-färbbar, wenn jeder endliche Teilgraph n-färbbar ist.\\
Färbungen: Fehlen im Ziegler-Skript!!\\
\section{Prädikatenlogik}
$"$Logik der ersten Stufe$"$\\
$"\forall_{x\in\mathbb{R}}\forall_{\delta>0}\exists_{\varepsilon > 0}\forall_{y\in\mathbb{R}}(|y-x|<\varepsilon \rightarrow |f(y)-f(x)|<\delta)"$ ~~~$"$f ist stetig$"$.
\subsection{Strukturen und Formeln}
\textbf{Definition}: Eine Sprache L besteht aus:\\
\begin{itemize}
\item einer Menge von Konstanten
\item einer Menge von Funktionszeichen
\item einer Menge vo Relationszeichen
\end{itemize}
Funktionen und Relationen haben eine feste $"$Stelligkeit$"$ (Zahl der erlaubten Argumente)$\geq$ 1\\ \\
\textbf{Beispiel}: Um Körper zu beschreiben, wählen wir Konstanten $\{0,1\}$, Funktionen $\{+,\cdot , -\}$ dabei haben $+,\cdot$ Stelligkeit 2 und - hat Stelligkeit 1. Für angeordnete Körper nehmen wir die Relation $\{<\}$ hinzu. Funktionen und Relationen werden in Präfix-Notation benutzt:\\
$$+ab~statt~a+b$$\\ \\
\textbf{Definition:} Sei L eine Sprache. Eine L-\underline{Struktur} $\mathfrak{A} =\{A,(Z^\mathfrak{A})_{Z\in L} \}$ wobei A eine Meinge ist und $Z^\mathfrak{A}$ je nach Typ von Z:\\
\begin{itemize}
\item $Z^\mathfrak{A} \in A$ falls Z Konstante ist
\item $Z^\mathfrak{A}$:$A^n\rightarrow A$ Abbildung, falls Z eim n-stelliges Funktionssymbol ist
\item $Z^\mathfrak{A} \subset A^n$ falls ein n-stelliges Relationssymbol ist
\end{itemize}
\textbf{Beispiel}: L wie oben, dann sind\\
$$(\mathbb{R},0_\mathbb{R},1_\mathbb{R},+_\mathbb{R},\cdot_\mathbb{R},-_\mathbb{R},<_\mathbb{R})$$\\
$$(\mathbb{Q},0_\mathbb{Q},1_\mathbb{Q},+\mathbb{Q},\cdot_\mathbb{Q},-_\mathbb{Q},<\mathbb{Q})$$\\
Beispiele von L-Strukturen.\\ \\
\textbf{Definition}: Ein \underline{Isomorphismus} zwischen L-Strukturen $\mathfrak{A}=(A,(Z^\mathfrak{A})_{Z\in L})$ und $\mathfrak{B}=(B,(Z^\mathfrak{A})_{Z\in L})$ ist eine bijektive Abbildung F:$A\rightarrow$B, die mit den Interpretationen der Zeichen aus L verträglich ist:\\
\begin{itemize}
\item F($Z^\mathfrak{A}$=Z$^\mathfrak{B}$ für Konstanten
\item F$\circ Z^\mathfrak{A}=Z^\mathfrak{B}\circ F^n:A^n\rightarrow B$ für Funktionssymbole
\item F$^n(Z^\mathfrak{A}=Z^\mathfrak{B}$ für Relationssymbole
\end{itemize}
Eine L-Antommorphismus von $\mathfrak{A}$ ist ein L-Isomorphismus von $\mathfrak{A}~nach~\mathfrak{A}$. Die Menge der L-Antomophismen bildet eine Gruppe Ant $\mathfrak{A}$.\\
Fixiere eine Menge von Variablen $v_0,v_1,v_2,...$\\ \\
\textbf{Definition}: Ein L-Term ist eine Zeichenfolge, die nach folgenden Regeln gebildet ist:\\
\begin{enumerate}
\item Jede Konstante ist ein L-Term
\item Jede Variable ist ein L-Term
\item Sei f ein n-stelliges Funktionssymbol und seien $t_1,...,t_n$ Terme, dann ist auch f$t_1,...,t_n$ ein Term. \\ \\
\end{enumerate}
\textbf{Beispiel:}$"+\cdot 2 \pi v_0"$ bedeutet $2\cdot \pi + v_0$\\
$"\cdot + v_0 v_1 \pi"$ bedeutet ($v_0+v_1)\cdot \pi$ \\ \\
\textbf{Lemma}: Für jeden term gilt genau einer der drei folgenden Fälle:
\begin{itemize}
\item t ist eine (eindeutige) Konstante 
\item t ist eine (eindeutige) Variable
\item t ist von der Form t=f $t_1...t_n$, dabei ist f ein n-stelliges Funktionssymbol und $t_1,..,t_n$ sind eindeutig bestimmte 
Terme. \\ \\
\end{itemize}
\textbf{Beweis}: Analog zu eindeutigen Lebarkeit aussagenlogischer Formeln. $\square$\\ \\
\textbf{Hilfsatz}: Kein Term ist ein echtes Anfangsstück eines anderen Termes.\\ \\
\textbf{Beweis}: Analog zur Aussagenlogik... $\square$\\  \\
Zur besseren Lesbarkeit schreiben wir manchmal f($t_1,...,t_n$) für f$t_1...t_n$\\

\textbf{Definition} Sei $L$ eine Sprache. Eine $L$-Formel ist
\begin{itemize}
\item $t_0 \circeq t_1$, wobei $t_0,t_1$ Terme. ''$t_0$ ist gleich $t_1$''
\item $Rt_0 \dots t_{n-1}$, $t_0, \dots ,t_{n-1}$ Terme, $R$ $n$-stelliges Relationssymbol
\item $\lnot F_0$, $F_0$ Formel
\item $(F_0 \land F_1)$, wobei $F_0, F_1$ Formeln
\item $\exists v\ F$, wobei $v$ Variable, $F$ Formel
\end{itemize}
Alle Formeln entstehen auf diese Weise.\medskip\\
\textbf{Beispiel:} $L=(0,f,+,<)$\\
$\forall v_0\forall v_1 (<0v_1 \rightarrow \exists v_2(<0v_2 \land ((<v_0+v_2v_3 \land <v_3+v_0v_2)\rightarrow (<fv_0+v_1fv_3 \land <fv_3+fv_0v_1))))$\\
% LaTeX kommt mit der Präfix-Notation nicht so gut klar. Hat jemand ne Idee, wie man das schöner machen kann?
''$f$ ist stetig'' mit $v_0=x, v_1=\epsilon, v_2=\delta, v_3=y$\medskip\\
\textbf{Abkürzungen:}\smallskip\\
\begin{tabular}{l|l}
Abkürzung & Bedeutung\\\hline
$\forall v\ F$ & $\lnot\exists v\ \lnot F$\\
$(F_0 \lor F_1)$ & $\lnot (\lnot F_0 \land \lnot F_1)$\\
$(F_0 \rightarrow F_1)$ & $(\lnot F_0 \lor F_1)$\\
$(F_0 \leftrightarrow F_1)$ & $((F_0 \rightarrow F_1) \land (F_1 \rightarrow F_0))$\\
$(F_0 \land \dots \land F_{n-1})$ & $(F_0 \land \dots \land (F_{n-2} \land F_{n-1})\dots)$\\
$(F_0 \lor \dots \lor F_{n-1})$ & $(F_0 \lor \dots \lor (F_{n-2} \lor F_{n-1})\dots)$\\
$\exists v_0 \dots v_{n-1}$ & $\exists v_0 \exists v_1 \dots \exists v_{n-1}$\\
$\forall v_0 \dots v_{n-1}$ & $\forall v_0 \forall v_1 \dots \forall v_{n-1}$
\end{tabular}\ \smallskip\\
Wenn Klammern weggelassen werden, gelten folgende Präzedenzregeln in absteigender Priorität:
\begin{enumerate}
\item $\exists,\forall,\lnot$
\item $\land$
\item $\lor$
\item $\rightarrow,\leftrightarrow$
\end{enumerate}
\textbf{Lemma:} $L$-Formeln sind eindeutig lesbar.\\
Das heißt: Auf jede Formel trifft genau einer der Fälle aus obiger Definition zu und die dort eventuell vorkommenden Terme und Teilformeln sind ebenfalls eindeutig.\smallskip\\
\textbf{Hilfssatz:} Keine Formel ist echtes Anfangsstück einer anderen Formel.\medskip\\
\textbf{Beweis:} Lemma und Hilfssatz durch Induktion über Länge der Formeln.\\
Fallunterscheidung nach erstem Symbol der Formel $F$:
\begin{itemize}
\item Falls erstes Symbol eine Variable, Konstante oder Funktion ist, muss $F$ vom Typ $F= t_0 \circeq t_1$ sein. Da Terme eindeutig lesbar sind, ist klar, wie $t_0$ und $t_1$ aufgebaut sind und wo $F$ aufhört.
\item Falls erstes Symbol ein Relationssymbol ist, gilt $F=Rt_0 \dots t_{n-1}$. Dabei legt $R$ die Stelligkeit $n$ fest und $t_0 \dots t_{n-1}$ sind eindeutig.
\item Falls erstes Symbol $\lnot$ ist, gilt $F= \lnot F_0$. Da $F_0$ kürzer als $F$ ist, ist $F_0$ nach IV eindeutig lesbar.
\item Falls erstes Symbol $($ ist, gilt $F \in \{(F_0 \land F_1),(F_0 \lor F_1),\dots\}$
\item Falls erstes Symbol ein Quantor ist, gilt $F=\exists v_1\ F_0$ bzw. $F=\forall v_1\ F_0$. Nach IV ist $F$ eindeutig lesbar.
\end{itemize}
$\square$
\subsection{Semantik}
\textbf{Definition:} Sei $\mathfrak{A}=(A,(Z^\mathfrak{A})_{Z \in L})$ eine $L$-Struktur. Eine \underline{Belegung} $\beta$ ist eine Abbildung
$$\beta \colon (v_0,v_1 \dots) \rightarrow A$$
wobei $(v_0,v_1 \dots)$ die Menge der Variablen ist.\medskip\\
\textbf{Definition:} Sei $\mathfrak{A}$ eine $L$-Struktur, $\beta$ eine Belegung.\\
Dann definiere für $L$-Terme $t$ rekursiv
\begin{itemize}
\item $t^\mathfrak{A}[\beta] = \beta(v_i)$ falls $t = v_i$
\item $t^\mathfrak{A}[\beta] = c^\mathfrak{A}$ falls $t = c$ Konstantensymbol
\item $t^\mathfrak{A}[\beta] = f^\mathfrak{A}(t_1^\mathfrak{A}[\beta], \dots ,t_{n-1}^\mathfrak{A}[\beta])$
\end{itemize}
\textbf{Beispiel:} $L=(e,0,\text{Inv})$ Sprache für Gruppen.\\
Sei $\mathbb{K}$ ein Körper. Dann kann man 2 $L$-Strukturen betrachten:\smallskip\\
$\mathfrak{A} = (\mathbb{K},0,+,-)$\\
$\mathfrak{L} = (\mathbb{K} \setminus \{u\},\land,\cdot,^{-1})$\smallskip\\
Sowie jede Menge möglicher anderer Strukturen.\medskip\\
\textbf{Definition:} Sei $\mathfrak{A}$ eine $L$-Struktur, $\beta$ eine Belegung.\\
Der Wahrheitswert von Formeln wird rekursiv wie folgt definiert:
\begin{itemize}
\item $\varphi= F_0 \circeq t_1$, dann ist $\varphi^\mfa[\beta]$ wahr, gdw. $t_0^\mfa[\beta] = t_1^\mfa[\beta] \in A$
\item $\varphi = Rt_0 \dots t_{n-1}$, dann ist $\varphi^\mfa[\beta]$ wahr, gdw. $(t_0^\mfa[\beta], \dots ,t_{n-1}^\mfa[\beta]) \in R^\mfa \subset A^n$
\item $\varphi = \lnot \varphi_0$, dann ist $\varphi^\mfa[\beta]$ wahr, gdw. $\varphi_0^\mfa[\beta]$ falsch ist
\item $\varphi = (\varphi_0 \land \varphi_1)$ dann ist $\varphi^\mfa[\beta]$ wahr, gdw. sowohl $\varphi_0^\mfa[\beta]$ als auch $\varphi_1^\mfa[\beta]$ wahr sind
\item $\varphi = \exists v_i\ \varphi_0$, dann ist $\varphi^\mfa[\beta]$ wahr, gdw. $\exists a \in A \colon \varphi_0^\mfa[\beta \frac{a}{v_i}]$, $(\beta \frac{a}{v})(v_j)=
\begin{cases}
a & i = j\\
\beta[v_j] & i \neq j\\
\end{cases} $
\end{itemize}\ \smallskip\\
Also hat $\exists$ in $\exists v_i\ \varphi_0$ den Gültigkeitsbereich $\varphi_0$. Dort wird die ''globale'' Bedeutung $\beta(v_i)\in A$ durch eine ''lokale'' Bedeutung $a \in A$ ersetzt.\medskip\\
Globale Variablen: frei\\
Lokale Variablen: gebunden\medskip\\
\textbf{Definition:}\\
Sei $v_i$ eine Variable, $\varphi$ eine $L$-Formel. Dann kommt $v_i$ in $\varphi$ \underline{frei} vor, wenn
\begin{itemize}
\item $\varphi = t_0 \circeq t_1$ und $v_i$ kommt in $t_0$ oder $t_1$ vor.
\item $\varphi = Rt_0 \dots t_{n-1}$ und $v_i$ kommt in einem der Terme vor.
\item $\varphi = \lnot \varphi_0$ und $v_i$ kommt in $\varphi_0$ frei vor.
\item $\varphi = \varphi_0 \land \varphi_1$ und $v_i$ kommt in $\varphi_0$ oder $\varphi_1$ frei vor.
\item $\varphi = \exists v_j\ \varphi_0$ und $v_i \neq v_j$ und $v_i$ kommt frei in $\varphi_0$ vor.
\end{itemize}
Insbesondere ist $v_j$ in $\exists v_j\ \varphi_0$ nicht frei.\medskip\\
\textbf{Lemma}\\
Seien $\beta,\gamma$ Belegungen in $L$-Struktur $\mfa$.
\begin{itemize}
\item Für einen $L$-Term $t$ gilt\\
$t^\mfa[\beta] = t^\mfa[\gamma]$, wenn $\beta$ und $\gamma$ auf allen Variablen, die in $t$ vorkommen übereinstimmen.
\item Für eine Formel $\varphi$ gilt\\
$\varphi^\mfa[\beta]$ gdw. $\varphi^\mfa[\gamma]$, falls $\beta$ und $\gamma$ auf allen Variablen, die in $\varphi$ frei vorkommen übereinstimmen.
\end{itemize}
\textbf{Beweis:}\\
Erste Aussage folgt induktiv über Länge von $t$.\smallskip\\
Zweite Aussage folgt ebenfalls induktiv:\\
Klar, falls $\varphi = t_0 \circeq t_1,\varphi = Rt_0 \dots t_{n-1},\varphi = (\varphi_0 \land \varphi_1),\varphi = \lnot \varphi_0$.\smallskip\\
Sei $\varphi = \exists v_i\ \varphi_0$ und $a \in A$ so, dass $\varphi_0^\mfa[\beta \frac{a}{v_i}]$ wahr ist. Dann stimmt $\gamma \frac{a}{v_i}$ mit $\beta \frac{a}{v_i}$ auf allen Variablen überein, die in $\varphi_0$ frei vorkommen.\\
Somit gilt auch $\varphi_0^\mfa[\gamma \frac{a}{v_i}]$, somit $\varphi^\mfa[\beta] \Rightarrow \varphi^\mfa[\gamma]$.\smallskip\\
Genau so: $\varphi^\mfa[\gamma] \Rightarrow \varphi^\mfa[\beta]$ \hfill $\square$\medskip\\
\textbf{Definition:}
Eine \underline{Aussage} ist eine $L$-Formel ohne freie Variablen.\medskip\\
Eine Aussage $\varphi$ gilt in $\mfa$ also unabhängig von Belegungen.\\
Dafür schreibt man $\mfa \vDash \varphi$, was man ''$\mfa$ ist Modell von $\varphi$'', ''$\varphi$ gilt in $\mfa$'' oder ''$\mfa$ erfüllt $\varphi$'' liest.\\
%Eingefügtes von Ilia

	\medskip 
	Terme: 
		$v_{i}, c, ft_{0} \dots t_{n-1}$ \\
	Formeln:  
		$t_0 \circeq t_1,\quad  Rt_0 \dots t_{n-1}, \quad $\textlnot$ \varphi, (\varphi_0 \wedge \varphi_1) $ \\
	Abkürzungen (z.B): 
		$(\varphi - \psi) $, $\varphi \Leftrightarrow \psi$\\
		$ \forall v, \psi \Leftrightarrow \lnot \exists v_i \lnot \psi $\\
	Sei $ \mfa = ((A, c^{\mfa}))_{c \in L},\quad f(\mfa)_{f \in L}, \quad  (R^{\mfa})_{R \in L} $eine L-Struktur, \\
	$\beta :  \lbrace Var \rbrace  \rightarrow$ A eine Belegung  \\
	\medskip 
	Dann erhalten wir 
		$t^{\mfa} \left[\beta \right] \in A $\\
		$\phi^\mfa \left[\beta \right] \in \lbrace wahr, falsch \rbrace $
		$\mfa \models \phi \left[ \beta\right], \mfa \nvDash \phi \left[ \beta \right]$ freie Variable \\
	Aussage = Formel ohne freie Variable \\
	$\phi$ Aussage:$ \mfa \models \phi \left[ \beta \right] \Leftrightarrow \mfa \models \left[ \gamma \right]$ unabhängig von der Belegung bedeutet:\\
		 $x_1, \dots x_n$ sind paarweise verschiedene Variablen (die durchaus gleiche Werte haben können) und in $t$ kommen diese Variablen vor.\\
	$\phi (x_1, x_n)$ bedeutet $x_1, \dots x_n$ wie oben, und in $\phi$ kommen nur $x_1, \dots x_n$ frei vor.\\
	Für $a_1 \dots a_n \in A$ schreibe $t^\mfa \left[ a_1 \dots a_n \right]$ bzw. $\phi^\mfa \left[ a_1 \dots a_n \right]$ für $t^\mfa \left[ \beta \right]$, $\phi^\mfa \left[\beta \right]$, wobei $\beta(x_i) = a_i$ für $i = 1, \dots, n$\\
	
	\underline{Wiederholung}:\\ 
		$F: \quad \mfa \rightarrow \mathfrak{b} \quad$ Isomorphismus von L-Strukturen, falls \\
		$F: \quad A \rightarrow B \quad $ bijektiv ist, und $F(c^\mfa)= c^{c^\mathfrak{b}}$ \\
		$F(f^{\mfa}(a_0,\dots a_{n-1})) = f^{\mathfrak{b}} (F(a_0), \dots ,F(a_{n-1}))$, \\
		$(a_0 \dots a_{n-1}) \in R^{\mfa} \Leftrightarrow  (F(a_0), \dots ,F(a_{n-1})) \in R^{\mathfrak{b}}$ für alle $c,f,R \in L$\\ 
	
	\textbf{Definition}: \\
		Zwei L-Strukturen heißen elementar äquivalent, wenn für alle L-Aussage gilt: \\
		$\mfa \models \phi \Leftrightarrow \mathfrak{b} \models \phi $\\
	
	\textbf{Lemma}: \\
		Isomorphe L-Strukturen sind elementar äquivalent.\\
	
	\textbf{Beweis}: \\ 
		Durch Rekursion, dazu brauchen wir zwischendurch auch Belegung $\beta$. \\
		Sei $F =  \mfa \rightarrow \mathfrak{b} $ ein Isomorphismus.\\
		Sei $\beta : \lbrace Var \rbrace \rightarrow A $ Belegung in $\mfa$, dann ist $F \circ \beta : \lbrace Var \rbrace \rightarrow B $ eine Belegung in $\mathfrak{b}$.\\
	
	\textit{1. Für Terme $t$ gilt:} \medskip \\
		$F(t^{\mfa}\left[ \beta \right]) = t^{\mathfrak{b}}\left[ F \circ \beta \right])$ \medskip \\
		$t= c: F(c^{\mfa}) = c^{\mathfrak{b}}$ \medskip \\
		$t: v: F(t^{\mfa}\left[ \beta \right]) = F(\beta (v))= (F \circ \beta)(v) = t^{\mathfrak{b}}\left[ f \circ \beta \right]$ \medskip \\
		$t: f_{t_0} , \dots f_{t_n}$ Hier unklar \medskip  \\ 		
		$F(t^{\mfa}\left[\beta \right]) = F(t^{\mfa}(t^{\mfa}_{0}\left[\beta \right], \dots , t^{\mfa}_{n-1}\left[\beta \right] )) $ \medskip \\
		$f^{\mathfrak{b}}(F(t^{\mfa}_{0}\left[\beta \right]), \dots , F(t^{\mfa}_{n-1}\left[\beta \right]))$ \medskip \\
		$f^{\mathfrak{b}}(t^{\mathfrak{b}}_{0}\left[ F\circ \beta \right], \dots t^{\mathfrak{b}}_{n-1}\left[ F\circ \beta \right]) = t^{\mathfrak{b}}\left[ F\circ \beta \right]$\medskip \\
	
	\textit{2. Für jede Formel $\phi$ gilt:}\\
		$\mfa \models \phi \left[\beta \right] \Leftrightarrow \mathfrak{b} \models \phi \left[F \circ \beta \right]$\medskip\\
		$\phi = t_0 \circeq t_1:$\medskip\\
		$\mfa \models \phi \left[\beta \right] \Leftrightarrow t^{\mathfrak{b}}_{0}\left[ \beta \right] = t^{\mathfrak{b}}_{1}\left[ \beta \right]$\medskip\\
		F injektiv: \medskip\\
			$\mfa \models \phi \left[ \beta \right] \Leftrightarrow t_0^{\mathfrak{b}}\left[ F \circ \beta \right] = F(t^{\mathfrak{b}}_{0}\left[ \beta \right]) = F(t^{\mathfrak{b}}_{1}\left[ \beta \right])$\medskip\\
			$t_1^{\mathfrak{b}}\left[ F \circ \ \beta \right] \Leftrightarrow \mathfrak{b} \models \phi \left[ F \circ \beta \right]$\bigskip \\
		$\phi = R_{t_0 \dots t_{n-1}}$ analog \medskip \\
		$\phi = \lnot \psi : \mfa \models \phi \left[ \beta \right] \Leftrightarrow \mfa \nvDash \psi \left[ \beta \right] $\medskip \\
		$\Leftrightarrow \mathfrak{b} \nvDash \psi \left[ F \circ \beta \right] \Leftrightarrow \mathfrak{b} \models \phi \left[F \circ \beta \right]$ \medskip\\
		$\phi = (\phi_0 \wedge \phi_1)$ analog \medskip \\
		$\phi = \exists v \phi :$ \bigskip\\
		
		\textbf{"$\Rightarrow$":} $\mfa \models \phi \left[\beta\right]$ \medskip \\
			$\Rightarrow \exists a \in A $ mit $\mfa \models \phi \left[ \beta \frac{a}{v}\right]$ \medskip\\
			Es gilt:\\
			$F \circ (\beta\frac{a}{v}) = (F \circ \beta) \frac{F(a)}{v}$ also folgt\medskip\\
			$\mfa \models \psi \left[(F \circ \beta) \frac{F(a)}{v}) \right]$ also \medskip\\
			$\mfa \models \phi \left[F \circ \beta \right]$ \medskip\\
		
		\textbf{"$\Leftarrow$":}\medskip\\
			$\mfa \models \psi \left[ F \circ \beta \right] \frac{b}{v}$ \medskip\\	
			$F$ surjektiv $\leftarrow \exists a \in A$ mit $F(a) = b$\medskip\\
			$\Rightarrow \mfa \models \psi \beta \left[ \beta \frac{a}{v} \right] \Rightarrow \mfa \models \psi \left[ \beta \right]$ \bigskip\\
			Bemerkung: Zwei endliche L-Strukturen $\mfa, \mathfrak{b}$ sind genau dann äquivalent, wenn sie äquivalent sind.\bigskip\\
			Es sei also A = $\lbrace a_1, \dots a_N$ endlich.\\
			Es reicht eube Aussage $\phi$ auszugeben mit $\mathfrak{b} \models \phi \Leftrightarrow \mfa $ und $\mathfrak{b}$ sind isomorph. \medskip\\ 
			
			Dazu sei\\
			$WIEDERLICHE FORMEL!$\medskip\\
			
			$\lceil $Diese Formel hat die Länge $O(N^{Max(k,L)})$ wobei k die maximale Stellenanzahl einen L-Symbols sei.$ \rceil$\\
			Gegeben $\mfa$ und $L$, braucht man $O(M^{c(N,L)})$ Schritte, um festzustellen, ob $\mathfrak{b}$ zu $\mfa$ isomorph ist.
			D.h.: $\mfa \cong \mathfrak{b}$ bei festen $\mfa$ ist ein P-Problem, bei $\mfa , \mathfrak{b}$ beliebig ein NP-Problem. \bigskip\bigskip\\
			
	\textbf{Substitution} \\
		Ersetze eine Variable x durch einen Term s\\\\
		Für Terme:\\
		$t = c: t \frac{s}{x} = c$\medskip\\
		$t = v: t \frac{s}{x}  \begin{cases}  s,&  v=x\\
											  v,&  v\neq x
							\end{cases} $\\\\
		$t = f_{t_{0}}\dots f_{t_{N-1}}: t\frac{s}{x} = f_{t_{0}} \frac{s}{x} \dots f_{t_{N-1}} \frac{s}{x}$
		
		\textbf{Lemma:} \\\\
			Es sei $t$ ein L-Term und $\mfa$ eine L-Struktur, x eine Variable, $\beta$ eine Belegung, dann gilt: \medskip\\
			$(t \frac{s}{x}) \left[ \beta \right] = t^{\mfa} \left[ \beta \frac{s^{\mfa}\left[\beta\right]}{x} \right]$ \medskip\\
		
		\textbf{Beweis:}\\\\
			Beweis durch Rekursion\\
			Entscheidender Schritt:\\
			$t = x: t\frac{s}{x} = s$, also $(t\frac{s}{x})^{\mfa}\left[ \beta \right] = s^{\mfa} \left[ \beta\right]$\medskip\\
			$= t^{\mfa}\left[\beta \frac{s^{\mfa}\left[\beta\right]}{x} \right]$\medskip\\
		
		\textit{Substitution für Formeln:}\\
			$\phi = t_0 \circeq t_1: \phi \frac{s}{x} = t_0 \frac{s}{x} =t_1 \frac{s}{x}$\medskip\\
			$\phi= R_{t_{0} \dots t_{N-1}}$ analog \medskip\\
			$\phi = \lnot \psi:\phi\frac{s}{x} = \lnot \psi \frac{s}{x}$\medskip\\
			$\phi = (\phi_0 \wedge \phi_1)$ analog \medskip \\
			$\phi \frac{s}{x}  \begin{cases}  \phi&  v=x\\
												\exists v\psi \frac{s}{x} &  v\neq x
										\end{cases} $\\\\
			Das heißt, nur freie Variablen werden ersetzt.\medskip\\
			Problem: Was ist mit Formeln wie:\medskip\\				
			$\exists v_0 Rv_0 v_1$ und $s=fv_0$?\medskip\\
			Dann ist $\phi \frac{s}{v_1} = \exists v_0 Rv_0 fv_0 \quad (v \neq x)$ \\
			Wenn $\beta_0 = a$, aber $\phi$ richtig ist für $v_0 = b_1,v_1 = c$ also $\phi \left[ v_0 \rightarrow a, v_1 \rightarrow c\right] = \exists v_0 R v_0 c$ , sei $f(a)=c$. \bigskip\\
			$\phi = \phi (v_0, v_1) = \exists v_0 \psi$ \\
			$\psi = \psi (v_0, v_1) = R v_0 v_1$ \\
			Seien $a,b,c$ paarweise verschieden\\
			$f(a) = c $\\
			$f(b) = d $\\
			und $R(b,c)$ sei wahr, alle anderen $R(\dots)$ falsch\\
			$\Rightarrow \phi \left[ a, c\right] = \exists v_0Rv_0fv_0$ und $\phi \left[ a, c \right]$  = $\exists v_0 R v_0 f v_0$ ist falsch, da man nicht $v_0 = b, R v_0 = c$ erreichen kann.\\\\\\
			
		\textbf{Definition:} \\
			Eine Variable $x$ ist frei für $s$ in $\phi$, wenn an den Stellen in $\phi$, an denen $x$ frei vorkommt keine Variable von $s$ im Gültigkeitsbereich eines Quantors liegt.\\\\
			
			Rekursiv:\medskip\\
			$\phi = t_0 \circeq t_1, \phi R t_0 \dots t_n-1 $ - $x$ frei für $s$ \medskip\\
			$\phi = \lnot \phi , \phi (\phi_0 \wedge \phi_1)$ - x frei für s in $\phi \Leftrightarrow x $frei für $s $in $\phi $bzw. in$ \phi_0$ und $\phi_1$ \medskip\\
			$\phi = \exists v \psi$ - x freu für s in $\phi \Leftrightarrow$\\ 
			$v = x$ oder ($v = x und v kommt nicht in s vor$)\medskip\\
			
		\textbf{Lemma:}\\
		Wenn x frei für s in $\phi$ ist und $\mfa$ eine L-Struktur, $\beta$ eine Belegung, dann gilt:\\
		$\mfa \models \phi \frac{s}{x} \left[ \beta \right] \Leftrightarrow \mfa \models \phi \left[ \beta \frac{s^{\mfa}\beta}{x} \right]$

		\textbf{Beweis:}\\\\
		Analog zum obigen. Entschedent ist der Schritt für $\phi = \exists v \phi$


%VL 16.12.13
\textbf{Wdh:}$\mathfrak{A},\mathfrak{B}$ elementar äquivalent $\Leftrightarrow (\mathfrak{A} \models \varphi \Leftrightarrow \mathfrak{B}\models \varphi$ für alle L-Aussagen $\varphi$)\\
Substitution: Es sei x eine Variable und s ein Term\begin{itemize}
\item Term t $\rightarrow$ t $\frac{s}{x}$ (nicht rekursives Ersetzen)
\item Formel $\varphi \to \frac{s}{x}$\\
Wichtig: nur \underline{frei} vorkommende  x werden durch s ersetzt.
\item x ist \underline{frei} für s in $\varphi \Leftrightarrow$ überall dort, wo x frei vorkommt, ist keine Variable von s in $\varphi$ gebunden.
\end{itemize}
\textbf{Lemma} (t$\frac{s}{x})^\mathfrak{A}[\beta]=t^\mathfrak{A}[\beta \frac{s^\mathfrak{A}[\beta]}{x}]$\\
Wenn x für s in $\varphi$ frei ist, gilt\\
$\mathfrak{A}\models(\varphi\frac{s}{x})[\beta]\Leftrightarrow \mathfrak{A}\models \varphi[\beta\frac{s^\mathfrak{A}[\beta]}{x}]$\\ \\ \\  \\
\textbf{Beweis (für Formeln):}Durch Induktion über die Länge von $\varphi$.
\begin{itemize}
\item $\varphi = t_0\doteq t_1$ Dann gilt:\\
$\mathfrak{A}\models(\varphi \frac{s}{x})[\beta] \Leftrightarrow \mathfrak{A}\models(t_0\frac{s}{x}\doteq t_1 \frac{s}{x})[\beta]\\
\Leftrightarrow(t_0\frac{s}{x}^\mathfrak{A} [\beta]=(t_1\frac{s}{x})^\mathfrak{A}[\beta]\in$ A\\
$\Leftrightarrow \mathfrak{A}\models(t_0 \doteq t_1)[\beta \frac{s^\mathfrak{A}[\beta]}{x}]\in $ A\\
$\Leftrightarrow \mathfrak{A}\models (t_0 \doteq t_1) [\beta \frac{s^\mathfrak{A}[\beta}]{x}]\Leftrightarrow \mathfrak{A}\models \varphi[\beta \frac{s^\mathfrak{A}[\beta]}{x}]$
\item $\varphi=R~t_0...t_{n-1}$: Analog.
\item $\varphi = \neg \psi: \mathfrak{A} \models (\varphi \frac{s}{x}) [\beta]\\
\Leftrightarrow \mathfrak{A}\not\models (\psi \frac{s}{x})[\beta]\\
\Leftrightarrow \mathfrak{A} \not\models \psi [\beta \frac{s^\mathfrak{A}[\beta]}{x}]\\
\Leftrightarrow \mathfrak{A} \models \varphi [\beta \frac{s^\mathfrak{A}[\beta]}{x}]$
\item $\varphi=(\varphi_0\land \varphi_1)$: Analog
\item Existenz quantifizierte Formeln:\\
1. Fall $\varphi = \exists x\phi$. x ist nicht mehr frei in $\psi$, also $\varphi \frac{s}{x}$=$\varphi$. Außerdem spielt der Wert von x in $\beta$ bzw. $\beta \frac{s^\mathfrak{A}[\beta]}{x}$ keine Rolle.\\
$\mathfrak{A}\models (\varphi \frac{s}{x})[\beta] \Leftrightarrow \mathfrak{A} \models \varphi [\beta]\\
\Leftrightarrow \mathfrak{A}\models \varphi [\beta \frac{s^\mathfrak{A}[\beta]}{x}]$.\\
2.Fall: $\varphi = \exists y\phi$, x,y verschiedene Variable und x sei frei in $\psi$. $\Rightarrow \varphi\frac{s}{x}=\exists y\psi \frac{s}{x}$\\
$\mathfrak{A} \models (\varphi\frac{s}{x})[\beta]\Leftrightarrow$ Es gibt b$\in$A, so dass: $\mathfrak{A}\models (\psi\frac{s}{x})[\beta\frac{b}{y}]\\
\Leftrightarrow \mathfrak{A}\models \psi [\beta\frac{b}{y}\frac{s^\mathfrak{A}[\beta\frac{b}{y}]}{x}]$ nach Induktion, mit der neuen Belegung $\beta \frac{b}{y}$. Nach Voraussetzung (x frei für s in $\varphi$) kommt y nicht in s vor, also gilt $s^\mathfrak{A}[\beta\frac{b}{y}]=s^\mathfrak{A}[\beta]\in $A. Somit:\\
...$\Leftrightarrow$ Es gibt b, so dass $\Leftrightarrow \mathfrak{A}\models \psi[\beta\frac{s^\mathfrak{A}[\beta]}{x}\frac{b}{y}]$\\
$\Leftrightarrow \mathfrak{A}\models(\exists y\psi)[\beta\frac{s^\mathfrak{A}[\beta]}{x}]\Leftrightarrow \mathfrak{A} \models \varphi[\beta \frac{s^\mathfrak{A}[\beta]}{x}]~~\square$
\end{itemize}
\subsection{Allgemeingültige Formeln}
Ziel: Begriff der Beweisbarkeit, zunnächst: Beweis. Bausteine dafür?\\
\textbf{Def.}: Eine Formel $\varphi$ $"$ heißt \underline{allgemeingültig}, kurz $"~\models\varphi$, wenn für alle L-Strukturen $\mathfrak{A}$ und alle Belegungen $\beta$ gilt:\\
$$\mathfrak{A}\models \varphi[\beta]$$\\
Äquivalent dazu gilt:\\
$$\mathfrak{A}\models \forall x_1... \forall x_n ~\varphi(x_1,...,x_n)$$
(Also sind $x_1,...,x_n$ mind. die freien Variablen von $\varphi$) für alle L-Strukturen $\mathfrak{A}$ \\
\textbf{Def.}: Eine \underline{Tautologie} entsteht, indem man eine aussagenlogische Tautologie F=F($A_0,...,A_{n-1}$) in den aussagenlogischen Variablen $A_0,...,A_{n-1}$ die Variablen $A_0,...,A_{n-1}$ durch L-Formeln $\varphi_0,...,\varphi_{n-1}$ ersetzt.\\
\textbf{Beispiel:} ($A_0 \lor \neg A_0$) ist eine aussagenlogische Tautologie, $\varphi=Rv_0v_1$ ist eine L-Formel (R zweistellige Relation)\\
$\Rightarrow (R_v0v_1 \lor \neg Rv_0v_1$) ist eine Tautologie.
Oder $\varphi_0 =(\exists v_0 Rv_0v_1 \land \neg v_1 \doteq c)$\\
$\Rightarrow ((\exists v_0R_0v_1 \land \neg v_1 \doteq c) v\neg \exists v_0Rv_0v_1 \land v_1 \doteq c)$ ist eine Tautologie\\ \\
\textbf{Lemma} Tautologien sind allgemeingültig.\\ \\
\textbf{Beweis}: klar. (Annmerkung des Authors: Natürlich darf er einfach trivial sagen ...)\\ \\
\textbf{Lemma (Gleichheitsaxiome)}: Die folgenden Formeln sind allgemeingültig, dabei stehen $x_0,x_1,...$ für beliebige Variablen.\\
Reflexivität: $x_0 \doteq x_0$\\
Symmetrie: $x_0\doteq x_1 \leftrightarrow x_1 \doteq x_0)$\\
Transitivität. $((x_o\doteq x_1 \land x_1 \doteq x_2)\rightarrow x_0 \doteq x_2)$\\
\begin{itemize}
\item Für jede k-stellige Relation R\\
$((x_0\doteq x_k \land ... \land x_{k-1} \doteq x_{2k-1})$\\
$\rightarrow (R_x0...x_{k-1} \leftrightarrow Rx_k ... x_{2k-1}))$\\
\item Für jede k-stellige Funktion f in L\\
$((x_0 \doteq x_k \land ... \land x_{k-1} \doteq x_{2k-1})$\\
$\rightarrow f x_0 ... x_{k-1} \doteq f x_k... x_{2k-1})$
\end{itemize}
\textbf{Beweis}: Die ersten drei Axiome gelten in allen Strukturen $\mathfrak{A}$, da $"="$ eine Äquivalenzrelation ist: z.B. seien a,b$\in$A, dann gilt a=b $\Leftrightarrow$ b= A.\\
Sei jetzt R eine K-stellige Relation.\\
$a_0=a_l,...,a_{k-1}=a_{2k-1} \Rightarrow (a_0,...,a_{k-1})=(a_k,...,a_{2k-1}) \in A^k$, somit ($a_0,...,a_{k-1})\in R^\mathfrak{A} \Leftrightarrow (a_k,...,a_{2k-1})\in R^\mathfrak{A}$\\
Analog für Funktionen. $\square$\\ \\
\textbf{Lemma} (Existenzquantoraxiome): Für alle Formeln $\varphi$, alle Variablen x und alle Terme s, so das x für s in $\varphi$ frei ist, ist allgemeingültig:\\
($\varphi\frac{s}{x}\rightarrow \exists x\varphi)$\\ \\
\textbf{Beweis}: Interessant sind nun $\mathfrak{A},\beta$, so dass $\mathfrak{A} \models (\varphi\frac{s}{x})[\beta]$. Nachddem Substitutionslemma gilt $\mathfrak{A}\models\varphi[\beta\frac{s^\mathfrak{A}[\beta]}{x}]$. Also gilt es a=s$^\mathfrak{A}[\beta]\in$ A, so dass $\varphi[\beta\frac{a}{x}]$. Also gilt $\mathfrak{A}\models(\exists x\varphi)[\beta].~~\square$\\ \\
\textbf{Lemma}(Modus Ponem): Seien $\varphi$ und $\psi$ Formeln. Aus $\mathfrak{A}\models\varphi$ und $\mathfrak{A}\models (\varphi \rightarrow \psi)$ folgt dann auch $\mathfrak{A}\models \psi$ \\ \\
\textbf{Beweis}: Klar $\square$\\ \\
Insbesondere: Seien $\varphi$ und ($\varphi\to \psi$) allgemeingültig, dann ist auch $psi$ allgemeingültig.\\ \\
\textbf{Lemma}(Existenzeinführung): Seien $\varphi,\psi$ Formeln und x eine Variable, die in $\psi$ nicht frei vorkommt. Wenn $\varphi \to \psi$ allgemeingültig ist, dann ist auch $(\exists x\varphi) \to \psi$ allgemeingültig.\\
Zur Errinerung: $A\neq \emptyset$ in jeder Struktur.\\ \\
\textbf{Beweis}: Interessant sind nur $\mathfrak{A},\beta$ mit $\mathfrak{A}\models (\exists x\varphi)[\beta]$. Also existiert a$\in$A mit $\mathfrak{A}\models \varphi[\beta\frac{a}{x}]$. Nach dem Modus Ponem gilt $\mathfrak{A}\models \psi [\beta\frac{a}{x}]$. Da x in $\psi$ nicht vorkommt, gitl $\mathfrak{A}\models \psi [\beta]$. Also auch $\mathfrak{A}\models(\exists x\varphi \rightarrow \psi)[\beta].~~\square$\\ \\
\textbf{Verschärfung des Lemmas}: Seien $\varphi, psi $ Formeln, x nicht frei in $\psi$.\\
Aus $\mathfrak{A}\models (\varphi \rightarrow  \psi)[\beta]$ folgt dann auch $\mathfrak{A} \models(\exists x\varphi \to \psi)[\beta]$.

\subsection{Der Gödelsche Vollständigkeitssatz}
\textbf{Definition}(Hilbertkalkül): Eine Formel $\phi$ in einer Sprache L heißt \underline{beweisbar}, wenn sie entweder ein Axiom der folgenden Typen (B1 - B3) ist, oder aus beweisbaren Formeln durch eine der folgenden Schlussregeln (B4, B5) ergibt.
\begin{itemize}
\item \textbf{B1 (Tautologie)} Es sei $f(a_o, ..., a_{n-1})$ eine allgemeingültige Formel der Aussagenlogik und $\phi_0,...,\phi_{n-1]}$ Formeln, dann ist $f(\phi_0, ...,  \phi_{n-1})$ eine \underline{Tautologie}.
\item \textbf{B2 (Gleichheitsaxiome)}
\begin{itemize} \item Äquivalenzaxiome, z.B. $(x = y \wedge y = z) \rightarrow y = z$
\item Kongruenzaxiome, z.B. $x = y \rightarrow f(x)\doteq f(y)$
\end{itemize}
\item \textbf{B3 (Existenzaxiome)} Es sei $x$ frei für $t$ in $\phi$, dann ist $\phi \frac{t}{x} \rightarrow \exists x \phi$ ein Existenzaxiom.
\item \textbf{B4 (Modus ponens)} Wenn $\phi \rightarrow \psi$ und $phi$ beweisbar ist, ist $\phi$ beweisbar.
\item \textbf{B5 (Existenzführung)} Es seien $\phi, \psi$ Formeln und $x$ sei nicht frei in $\psi$. Wenn $\phi \rightarrow \psi$ beweisbar ist, dann ist auch $\exists x \phi \rightarrow \psi$ beweisbar.
\end{itemize}
Ein \underline{Beweis} von $\phi$ ist also ein Tupel von Formeln $phi_0, ..., \phi_n$ mit $\phi_n = \psi$, so dass für alle i die Formel $\phi_i$ ein Axiom vom Typ B1 - B3 ist, oder aus den Formeln $\phi_1$ bzw. $\phi_j$ und $\phi_k$ und $j, k < 1$ durch einen Schluss vom Typ B4 oder B5 hervorgeht.\\\\
\textbf{Satz (Gödelscher Vollständigkeitssatz 1)}: Eine Formel $\phi$ in einer Sprache L ist genau dann beweisbar, wenn sie allgemeingültig ist:\\
\begin{center}
$\vdash \phi \Leftrightarrow \models \phi$
\end{center}
\newpage
\textbf{Notation}: "$t \subset \phi$" bedeutet: "$\phi$ ist in der Sprache L beweisbar." \\ Die Richtung "$\Leftarrow$" folgt aus Abschnitt 2.3.\\
\textbf{Lemma}: (abgeleitete Axiome und Schlussregeln).
\begin{itemize}
\item \textbf{1. Aussagenlogik}: Seien $\phi_1, ..., \phi_n$ beweisbar und $(\phi_1 \wedge ... \wedge \phi_n) \rightarrow \psi$ eine Tautologie, dann ist $\psi$ beweisbar.
\item \textbf{2. All-Axiom}: Sei $x$ frei für $t$ in $\phi$, dann ist $\forall x \phi \rightarrow \phi \frac{t}{x}$ beweisbar.
\item \textbf{3. All-Einführung}: Es sei $x$ frei in $\phi$. Wenn $\phi \rightarrow \psi$ beweisbar ist, dann auch $\phi \rightarrow \forall x \psi$.
\end{itemize}
\textbf{Beweis}: Zu 1.) $(\phi_1 \wedge ... \wedge \phi_n) \rightarrow \psi$ ist eine andere Schreibweise für $\neg \phi_1 \vee ... \vee \neg \phi_n \vee \phi_i$, bzw. für $\phi_1 \rightarrow (\phi_2 ... \rightarrow (\phi_n \rightarrow \psi) ... )$. Insbesondere ist $(\phi_1 \wedge ... \wedge \phi_n) \rightarrow \psi) \rightarrow (\phi \rightarrow (\phi_2 \rightarrow ... \rightarrow (\phi_n \rightarrow \psi) ... ))$ eine Tautologie, somit auch $\phi_1 \rightarrow (\phi_2 \rightarrow ... \rightarrow \phi_n \rightarrow \psi) ... )$\\
Mit Modus ponens sind dann auch $\phi_2 \rightarrow ... \rightarrow (\phi_n \rightarrow \psi)$ und $\phi_n \rightarrow \psi$ beweisbar.\\
Zu 2. $\forall x \phi \rightarrow \phi \frac{t}{x}$ ist "Makro" für \sout{$\neg ( \neg$}$\exists x \neg \phi$ \sout{$)$}$\vee \phi \frac{t}{x}$ und das ist äquivalent zu $\neg \phi \frac{t}{x} \rightarrow \exists x \neg \phi$. Da $x$ für $t$ frei in $\phi$ ist, ist $x$ für $t$ frei in $\exists x \neg \phi \neg \phi$, und $\neg \phi \frac{t}{x} \rightarrow$ ist ein Existenzaxiom. Daraus folgt $\forall x \phi \rightarrow \phi \frac{t}{x}$ mit Aussagenlogik.\\
Zu 3. Es ist $\phi \rightarrow \forall x \phi$ äquivalent zu $\neg \phi \vee \neg \exists x \neg \psi$ und $x$ ist nicht frei in $\neg \phi$. Aus $\phi \rightarrow \psi$ folgt $\neg \psi \rightarrow \neg \phi$ mit Aussagenlogik und daraus $\exists x \neg \psi \rightarrow \neg \phi$ mit Existenzeinführung. Mit Aussagenlogik folgt $\phi \rightarrow \neg \exists x \neg \psi$, also $\phi \rightarrow \forall x \psi$. $\square$\\\\
\textbf{Beispiel}: Beweise $\exists x \forall y R x y \rightarrow \forall y \exists x R x y$.
\begin{itemize}
\item{1)} $\forall y R x y \rightarrow R x y$ (All-Axiom)
\item{2)} $R x y \rightarrow \exists x R x y$ (Existenz-Axiom)
\item{3)} $\forall y R x y \rightarrow \exists x R y y$ (Aussagenlogik aus 1, 2)
\item{4)} $\exists x \forall y R x y \rightarrow \exists x R x y$ (Existenz-Einführung)
\item{5)} $\exists x \forall y R x y \rightarrow \forall y \exists x R x y$ (All-Einführung)
\end{itemize}
All-Einführung: Sei $\phi = T$ (z.B. $\phi = (x = y))$. Dann ist $\phi \rightarrow \psi$ äquivalent zu $\psi$. Also ist mit $\psi$ auch $\forall x \psi$ beweisbar. Um den Gödelschen Vollständigkeitssatz zu beweisen, reicht es, Aussagen zu betrachten.\\
\textbf{Lemma}: Es sei $\phi = \phi (x_1, ..., x_n)$ eine L-Formel. Es sei $L = \{ c_1, ..., c_n \}$ distjunkt zu L: Dann gilt $\forall_L \phi (x_1, ..., x_n) \Leftrightarrow \vdash_L \vee c \phi (c_1, ..., c_n).$
\textbf{Beweis}: "$\Rightarrow$" Mit der All-Einführung folgt: Wenn $\phi (x_1, ..., x_n)$ beweisbar ist, dann ist $\forall x ... \forall y_n \phi (x_1, ..., x_n)$ beweisbar in L- Dann ist der zugehörige L-Beweis auch ein ($L \cup C$)-Beweis, in dem aber $c_1, ..., c_n$ (noch) nicht vorkommen. Folgt $\forall x_1 ... \forall y_n \phi (x_1, ..., y_n) \rightarrow \phi (c_1, ..., c_n)$ wird Modus ponens gewählt "$t_{L \cup C} \phi (c_1, ..., c_n)$.\newpage
"$\Leftarrow$"$ $Es sei $B = (\phi_1, ..., \phi_n = \phi (c_1, ..., c_n))$ ein $L \cup C$-Beweis von $\phi (c_1, ..., c_n$. Es seien $y_1, ..., y_n$ Variablen, die noch nicht vorkommen. Ersetzen von $c_1, ..., c_n$ durch $y_1, ..., y_n$ liefert einen L-Beweis von $\phi(y_1, ..., y_n)$. All-Einführung liefert einen L-Beweis $\forall y_1 ... \forall y_n \phi (y_1, ..., y_n)$. Mit All-Axiomen und Modus ponens folgt $\phi (x_1, ..., x_n)$. $ \square$\\
Weitere Schritte im  Beweis des Vollständigkeitssatzes.
\begin{itemize}
\item{1)} Erweitere $\{ \neg \phi \}$
\item{2)} Vervollständige Theorie
\item{3)} Versuche ein Model für die Theorie zu finden
\end{itemize}
\textbf{Definition}: Eine \underline{L-Theorie} ist eine Menge von L-Aussagen. Eine L-Struktur $a$ ist ein Model von $T$, $a \models T$, wenn alle Aussagen aus $T$ im Modell $a$ gelten. Eine Aussage $\psi$ \underline{folgt logisch aus $T$}, $T \models \psi$, wenn $\psi$ in allen Modellen von $T$ gilt.\\\\
\textbf{Definition}: Eine Formel $\phi$ ist in $T$ \underline{beweisbar} in der Sprache L, kurz $T \vdash_L \psi$, wenn es Aussagen $\phi_1, ..., \phi_n \in T$ gibt, sodass $\vdash_L ((\phi_1 \wedge ... \wedge \phi_n) \rightarrow \phi)$\\
Eine Theorie $T$ ist \underline{widersprüchlich}, wenn $T \vdash_L \bot$, das heißt, wenn $phi_1, ..., \phi_n \in T$ existieren mit $\vdash_L \neg (\phi_1 \wedge ... T$ widerspruchsfrei $\Leftrightarrow T$ ist nicht mehr widersprüchlich. Wir kennen $T$ vollständig, wenn für jede $L$-Aussage $\psi$ gilt: $\psi \in T$ oder $\neg \psi \in T$. Eine Formel $\psi$ heißt in $T$ \underline{konsistent}, wenn $\models \cup \{ \psi \}$ widerspruchsfrei ist.\\\\
\textbf{Satz (Gödelscher Vollständigkeitssatz 2)}: Eine Theorie ist genau dann widerspruchsfrei, wenn sie ein Modell besitzt.\\\\
\textbf{Behauptung)}: "Gödel 2" impliziert "Gödel 1". Zu Gödel 1 sei $\phi$ eine $L$-Aussage. Klar: $\vdash_L \phi \Rightarrow \models \phi$ (Abs. 2.3). Wir nehmen an, dass $\models \phi$. Also existiert kein Modell für $\neg \phi$. Nach Gödel 2 ist $\{ \neg \phi \}$ widersprüchlich $\vdash_L \neg \neg \phi$. Mit Aussagenlogik folgt $\vdash_L \phi$. $\square$
\textbf{Beweis}: Es sei $a$ eine $L$-Struktur. Das vollständige Diagramm von $a$ ist die Theorie $\vdash_a = \{ \phi | \phi$ ist L-Aussage und $a \models \phi \}$. Dann ist $\vdash_a$ vollständig, denn in $a$ gilt entweder $\phi$ oder $\neg \phi$.\\\\
\textbf{Definition}: Es sei $L$ eine Sprache $C$ eine Menge von Konstanten. Dann heißt $T$ eine \underline{Henkin-Theorie} zu $L \cup C$, wenn für alle $L \cup C$-Formeln $\phi(x)$ eine Konstante $c \in C$ entsteht, sodass $\exists x \phi(x) \rightarrow \phi \frac {c}{x}$
\end{document}
