\documentclass[a4paper]{scrartcl}
\usepackage[utf8]{inputenc}
\usepackage[ngerman]{babel}
\usepackage{textcomp}
\usepackage{amsmath}

% zus�tzliche mathematische Symbole, AMS=American Mathematical Society 
\usepackage{amssymb}

% f�rs Einbinden von Graphiken
\usepackage{graphicx}

% f�r Namen etc. in Kopf- oder Fu�zeile
\usepackage{fancyhdr}

% erlaubt benutzerdefinierte Kopfzeilen 
\pagestyle{fancy}

% Definition der Kopfzeile
\lhead{
\begin{tabular}{ll}
Logik für Studierende der Informatik
\end{tabular}
}
\chead{}
\rhead{\today{}}
\lfoot{}
\cfoot{Seite \thepage}
\rfoot{} 

\begin{document}

\section*{\underline{1. Aussagenlogik}}

\subsection*{\underline{1. 1. Grundbegriffe}}
\textbf{Defintion:} (Syntax) Eine aussagenlogische Formel ist eine Zeichenkette, die sich aus den Variablen $A_0, A_1,$ ... und den Symbolen $(, ), \wedge, \vee, \neg$ zusammensetzt und folgende Regeln einhält:
\begin{itemize}
\item Variablen sind Formeln 
\item Wenn $F_1$ und $F_2$ Formeln sind, dann ist $(F_1 \wedge F_2)$ eine Formel (\textit{Konjunktion})
\item Wenn $F_1$ und $F_2$ Formeln sind, dann ist auch $(F_1 \vee F_2)$ eine Formel (\textit{Disjunktion})
\item Wenn F eine Formel ist, ist auch $\neg F$ eine Formel (\textit{Negation})
\end{itemize}
Alle Formeln enstehen auf diese Weise.\\
\textbf{Beispiel:} $(\neg A_0 \wedge A_1)$\\
\textbf{Lemma:} (Eindeutige Lesbarkeit) Jede Formel ist
\begin{itemize}
\item eine Variable
\item eine Konjunktion $(F_1 \wedge F_2)$
\item eine Disjunktion $(F_1 \vee F_2)$
\item eine Negation $\neg F$
\end{itemize}
Es tritt immer genau einer dieser Fälle ein, und ggf sind $F_1$ und $F_2$, bzw. $F$ eindeutig bestimmt.\\
\textbf{Beispiel:} Wir bezeichnen das erste Zeichen der gegebenen Formel. Es treten die Fälle auf:
\begin{itemize}
\item es ist eine Variable $A_i$. Dann ist $A_i$ nach Definition die ganze Formel.
\item es ist das Zeichen ''$\neg$'', dann ist nach Definition unserer Formel $\neg F$. $F$ ist also der Rest unserer Zeichenkette.
\item es ist ''$($''. Nach Definition ist unsere Formel vom Typ $(F_1 \wedge F_2)$ / $(F_1 \vee F_2)$.
\end{itemize}
Wir nehmen an, dass wir die Formel außerdem auch als  $(F'_1 \wedge F'_2)$ / $(F'_1 \vee F'_2)$ lesen können. Dann ist entweder $F'_1$ Anfangsstück von $F_1$ oder umgekehrt. Nach dem folgenden Hilfssatz gilt $F_1 = F'_1$ und das unmittelbar folgende Zeichen muss ''$\wedge$'' oder ''$\vee$'' sein und legt den Typ der Formel fest. Wenn man eine Formel nach obigem Lemma zerlegt hat und die Formel keine Variable war, dann wendet man das Lemma anschie{\''ss}end weiter auf $F_1$ und $F_2$ bzw. $F$ an bis die Formel komplett zerlegt wurde.\\
\textbf{Hilfssatz:} Kein echtes Anfangsstück einer Formel ist selbst eine Formel. Mit ''\textit{echtem Anfangsstück}'' meinen wir ein Anfangsstück, das nicht die ganze Formel ist.\\
\textbf{Beispiel:} $(F_1$ ist Anfangsstück von $(F_1 \wedge F_2)$.\\
\textbf{Beweis:} Durch Induktion über die Länge der Formel. Sei $F$ Formel der Länge 1, dann hat $F$ das echte Anfangsstück '' '' (\textit{leere Formel}), aber das ist keine Formel.\\ 

\newpage

Wir nehmen jetzt an, dass der Hilfssatz für alle kürzeren Formeln bewiesen werden kann. Wir wenden eine Fallunterscheidung nach dem ersten Buchstaben an.
\begin{itemize}
\item Variable $A_1$: Dann hat $F$ die Länge 1.
\item ''$\neg$'': Nach Definition gilt: $F = \neg F_1$. Sei $F'$ echtes Anfangsstück von $F_1$, dann folgt $F' = \neg F_1$, und $F'_1$ ist echtes Anfangsstück von $F_1$. Nach Induktionsvoraussetzung ist aber kein echtes Anfangsstück von $F_1$ eine Formel. Also kann $F'$ keine Formel sein.
\item ''$($'': Wie im obigen Beweis gilt dann $F = (F_1 \wedge F_2)$ / $(F_1 \vee F_2)$. Sei $F'$ echtes Anfangsstück von $F$, dann folgt $F' = (F_1 \wedge F_2)$ / $(F_1 \vee F_2)$. Da $F_1$ und $F_2$ kürzer als $F$ sind und $F_1$ Anfangsstück von $F'_1$ oder $F'_1$ Anfangsstück von $F_1$ ist, muss $F_1 = F'_1$ gelten. Dann muss $F'_2)$ ein Anfangsstück von $F_2)$ sein, also $F'_2$ ein Anfangsstück von $F_2$. Nach Induktionsvoraussetzung kann $F'_2$ kein echtes Anfangsstück gewesen sein.
\end{itemize}
Damit wäre der Hauptsatz bewiesen. $\square$\\
\textbf{Definition:} Es sei $\mathcal{D}$ eine Menge von Variablen, dann ist eine Belegung eine Abbildung $\mathcal{A}$:
\begin{center}
$\mathcal{A}: \mathcal{D} \rightarrow \{0, 1\}$
\end{center}
Dabei stehen 0, 1 für die Wahrheitswerte ''\textit{falsch}'' oder ''\textit{wahr}''. Sei also $A_3 \in \mathcal{D}$ und $\mathcal{A}(A_3) = 1$, dann hat $A_3$ unter der Belegung $\mathcal{A}$ den Wahrheitswert 1 (''\textit{wahr}''). Wir definieren Verknüpfungen von Wahrheitswerten:

\[ w_1 \wedge w_2 = \left\{
  \begin{array}{l l}
    1 & \quad \text{$w_1 = 1$ und $w_2 = 1$}\\
    0 & \quad \text{$w_1 = 0$ oder $w_2 = 0$}
  \end{array} \right.\]
\[ w_1 \vee w_2 = \left\{
  \begin{array}{l l}
    1 & \quad \text{$w_1 = 1$ oder $w_2 = 1$}\\
    0 & \quad \text{$w_1 = 0$ und $w_2 = 0$}
  \end{array} \right.\]
\[ \neg w_1 = \left\{
  \begin{array}{l l}
    1 & \quad \text{$w = 0$}\\
    0 & \quad \text{$w = 1$}
  \end{array} \right.\]
\textbf{Definition:} (Semantik) Sei $\mathcal{A}$ eine Belegung der Variablen einer Formel $F$, dann hat $F$ den Wahrheitswert $\mathcal{A}(F)$ von $F$:
\begin{itemize}
\item $\mathcal{A}(A_1)$, falls $F = A_1$ Variable ist
\item $\mathcal{A}(F_1) \wedge \mathcal{A}(F_2)$, falls $F = (F_1 \wedge F_2)$
\item $\mathcal{A}(F_1) \vee \mathcal{A}(F_2)$, falls $F = (F_1 \vee F_2)$
\item $\neg \mathcal{A}(F')$, falls $F = \neg F'$
\end{itemize}

\newpage

Wegen des obigen Lemma ist $\mathcal{A}(F)$ dadurch wohldefiniert. Wir verwenden folgende Abkürzungen:
\begin{itemize}
\item $(F_1 \rightarrow F_2)$ für $(\neg F_1 \vee F_2)$
\item $(F_1 \Leftrightarrow F_2)$ für $((F_1 \rightarrow F_2) \wedge (F_2 \rightarrow F_1))$
\item $\bigwedge_{i < n}{F_i}$ für ($F_1 \wedge ... (F{n-2} \wedge F{n-1}) ... )$
\item $\bigvee_{i < n}{F_i}$ für ($F_1 \vee ... (F{n-2} \vee F{n-1}) ... )$
\item $\top$ (''\textit{wahr}'') für $(\neg A_0 \vee A_0)$
\item $\bot$ (''\textit{falsch}'') für $(\neg A_0 \wedge A_0)$
\end{itemize}
\textbf{Konventionen:} Wir können $\top$, $\bot$ auch als ''\textit{nicht zusammengesetzte}'' Formeln betrachten
\begin{itemize}
\item $\bigwedge\limits_{i < 0}{F_i} = \top$
\item $\bigvee\limits_{i < 0}{F_i} = \bot$
\end{itemize}
Außerdem können wir $(F_1 \vee F_2)$ als Abkürzungen für $\neg(\neg F_1 \wedge \neg F_2)$ auffassen. Gelegentlich lassen wir Klammern weg, wenn keine Missverständnisse auftreten. Unsere Abkürzungen haben genau wie alle Formeln Wahrheitswerte unter vorgegebenen Bedingungen.\\
\begin{center}
$\mathcal{A}(\top) = 1, \mathcal{\bot} = 0,$ usw.
\end{center}
Sei $\mathcal(A)$ eine Belegung der Variablen von $F$. Wenn $\mathcal{A}(F) = 1$, sagen wir, ''\textit{$\mathcal{A}$ erfüllt $F$}'', ''\textit{$\mathcal{A}$ ist Modell von $F$}'', ''\textit{$\mathcal{A} \models F$}''.\\
\textbf{Definition:} Eine Formel heißt ''\textit{allgemeingültig}'' oder ''\textit{Tautologie}'', wenn $\mathcal{A}(F) = 1$ für alle möglichen Belegungen gilt. Eine Formel $F$ heißt ''\textit{erfüllbar}'', wenn es eine Belegung $\mathcal{A}$ mit $\mathcal{A}(F) = 1$ gibt.\\
\textbf{Beispiel:}
\begin{itemize}
\item $\top = (\neg A_0 \vee A_0)$ ist allgemeingültig, $A_0$ ist erfüllbar.
\item $\bot = (\neg A_0 \wedge A_0)$ ist nicht erfüllbar. $A_0$ ist nicht allgemeingültig.
\end{itemize}
\textbf{Lemma:} 
\begin{itemize}
\item $F$ ist genau dann eine Tautologie, wenn $\neg F$ nicht erfüllbar ist.
\item $F$ ist erfüllbar genau dann, wenn $\neg F$ keine Tautologie ist.
\end{itemize}
\end{document}
